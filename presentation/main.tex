\documentclass{beamer}

\usetheme{Madrid} 
\setbeamertemplate{section in toc}{\inserttocsectionnumber.~\inserttocsection}
\setbeamertemplate{subsection in toc}{
  \hspace{5mm}\inserttocsectionnumber.\inserttocsubsectionnumber.~\inserttocsubsection
  \breakhere}
\setbeamertemplate{subsubsection in toc}{}
\setbeamertemplate{navigation symbols}{} % To remove the navigation symbols from the bottom of all slides uncomment this line


\AtBeginSubsection[]{
  \begin{frame}
   %\frametitle{\insertsectionhead}
    
    \begin{center}
      \huge{\insertsectionhead}
    \end{center}
    \vspace{5mm}
    \begin{center}
      \begin{beamercolorbox}[sep=8pt,center,shadow=true]{title}
        \usebeamerfont{title}
        \insertsubsectionhead\par%\\
      \end{beamercolorbox}
    \end{center}
  \end{frame}
}
\defbeamertemplate*{footline}{Dan P theme}
{
  \leavevmode%
  \hbox{%
  \begin{beamercolorbox}[wd=.25\paperwidth,ht=2.25ex,dp=1ex,center]{author in head/foot}%
    \usebeamerfont{author in head/foot}\insertshortauthor\expandafter~~(\insertshortinstitute)
  \end{beamercolorbox}%
  \begin{beamercolorbox}[wd=.5\paperwidth,ht=2.25ex,dp=1ex,center]{title in head/foot}%
    \usebeamerfont{title in head/foot}\insertshorttitle
  \end{beamercolorbox}%
  \begin{beamercolorbox}[wd=.25\paperwidth,ht=2.25ex,dp=1ex,right]{date in head/foot}%
    \usebeamerfont{date in head/foot}\insertshortdate{}\hspace*{2em}
\insertframenumber{} / \inserttotalframenumber\hspace*{2ex} 
  \end{beamercolorbox}}%
  \vskip0pt%
}
\newcommand{\ib}[1]{\item \textbf{#1} }

\usepackage{graphicx} % Allows including images
\usepackage{booktabs} % Allows the use of \toprule, \midrule and \bottomrule in tables
\usepackage{import}
\usepackage{eurosym}
\usepackage[utf8]{inputenc}
\usepackage[spanish]{babel}
\usepackage{media9}    
\usepackage{etoolbox} 
\usepackage{tikz}
\usepackage{subcaption}
\usepackage{listings}
\usepackage[sort&compress,sectionbib]{natbib}
\bibliographystyle{IEEEtranN}

\newcommand{\frameitc}[4]{
  \begin{frame}{#1}
    \begin{center}
      \includegraphics[width=0.9\linewidth,height=0.7
      \textheight,keepaspectratio]{img/#2}\\
      #3~\cite{#4}.
    \end{center}
  \end{frame}
}

\newcommand{\frameitcp}[5]{
  \begin{frame}{#1}
    \begin{center}
      \includegraphics[width=#5\linewidth,height=0.7
      \textheight,keepaspectratio]{img/#2}\\
      #3~\cite{#4}.
    \end{center}
  \end{frame}
}

\newcommand{\frameit}[3]{
  \begin{frame}{#1}
    \begin{center}
      \includegraphics[width=0.9\linewidth,height=0.7
      \textheight,keepaspectratio]{img/#2}\\
      #3.
    \end{center}
  \end{frame}
}

\newcommand{\framei}[2]{
  \begin{frame}{#1}
    \begin{center}
      \includegraphics[width=0.9\linewidth,height=0.7
      \textheight,keepaspectratio]{img/#2}
    \end{center}
  \end{frame}
}

\newcommand{\framet}[2]{
  \begin{frame}{#1}
    #2
  \end{frame}
}

\title[Towards a Decentralized Publication System]{\large{Towards a Decentralized Publication System: A Proposal Using Blockchain and P2P Technologies}}

\author[V. Jacynycz]{Viktor Jacynycz García}

\institute[UCM] % Your institution as it will appear on the bottom of every slide, may be shorthand to save space
{
  Máster en ingeniería informática \\
  Master’s degree in Computer Science Engineering \\
  Facultad de informática\\
  Universidad Complutense de Madrid \\ % Your institution for the title page
  \medskip
  \textit{vsjg@ucm.es} % Your email address
}
\date{} % Date, can be changed to a custom date

\setcitestyle{numbers,square}
% ~\citestyle{nature}
\bibliographystyle{unsrtnat}

\begin{document}
\begin{frame}
\titlepage % Print the title page as the first slide
\end{frame}

\begin{frame}
\frametitle{Index}
%\tableofcontents
    \begin{minipage}{0.7\textwidth}
       \linespread{1.1}
       \tableofcontents[hideallsubsections] 
       \vspace{10mm}
    \end{minipage}
\end{frame}

\section{Introduction}
\framei{Introduction}{Research_Journals.jpg}
\framei{Introduction}{publishers_logos.png}
\framei{Introduction}{ScientificReview.jpg}
\framei{Introduction}{decsci.jpg}

\framet{Objectives}{
This work proposes three main objectives:
\vspace{5mm}
\begin{enumerate}
    \item Create a decentralized platform for science publishing.
    \vspace{5mm}
    \item Create a reputation system for reviewers.
    \vspace{5mm}
    \item Analyze the platform and compare it with the current sytems.
\end{enumerate}
}


\section{Background}
\subsection{Socio-cultural Background}
\subsubsection{Publication systems}
\frameitc{Publication systems}{Philosophical_Transactions_Volume_1_frontispiece.jpg}{The first \emph{journals} began to emerge in 1665}{kronick1976history}
\framei{Publication systems}{publishers.PNG}

\frameitc{Publication methods}{publishing.png}{Formerly, journal issues were printed by publishers}{spier2002history}

\frameitc{Peer review process (\emph{1972})}{peerreview.png}{The peer review process determines if a paper is published in a \emph{journal}}{spier2002history}

\frameitc{Publishers nowadays}{publishers2.png}{In the internet era, were creating a copy of a document has not cost, the role of traditional publishers could be questioned}{lariviere2015oligopoly}

\framet{Alternative publication systems}{
  \begin{itemize}
    \ib{Preprints:} Papers published before the peer-review process~\cite{shuai2012scientific}.
    \ib{Mega-journals:} A combination of varioues \emph{journals} to encourage open access publishing~\cite{binfield2013open}
    \ib{Countinuous publication:} A methodology followed by some online \emph{journals} that consists in publishing papers inmediatly after they pass the peer-review process~\cite{anderton2013continuous}.
  \end{itemize}
}

\frameitc{Reputation systems}{reputation.png}{Reputation systems offers the possibility to trust unknown individuals}{resnick2000reputation}

\frameit{Sistemas de reputación}{reputation2.png}{Many of the services we use today have a reputation system}

\frameit{Sistema de rating}{amazon.jpg}{Amazon, Google and many other companies have a 5-star scoring system}
  
\frameit{Sistema de votación}{se.png}{Stack exchange simplifies the voting with only two possibilities, like or dislike}
  
\frameitc{Peer review and publishing software}{ems.png}{Some of the state-of-the-art software for today's scientific publication process}{azar2006academic,paulo2011aems,Lev2016,rajpert2016rewarding,willinsky2005open}

\subsection{Technical Background}
\subsubsection{Distributed Networks}
\frameitc{Network architectures}{architectures.png}{Different configurations of network architectures}{baran1964distributed}

\frameitcp{Bitcoin}{btci.png}{Bitcoin is the first distributed cryptocurrency}{nakamoto2008bitcoin}{0.7}

\frameitc{Transation based system}{transaction.png}{Bitcoin stores all the transaction between users}{antonopoulos2014mastering}

\frameit{Bitcoin Addresses}{btcaddress.png}{Bitcoin users are identified inside the network using a randomly generated string called ``address''}

\frameitc{Blockchain}{blockchain.png}{Blockchain's state is saved in an encrypted database called the blockchain}{nakamoto2008bitcoin}

\frameitc{Ethereum}{EthereumLogo1200.jpg}{Ethereum uses the blockchain technology to deploy and execute small distributed programs}{nakamoto2008bitcoin}

\frameitc{Smart contracts}{sc.png}{Los contratos inteligentes son
  pequeños fragmentos de código que cambian el estado de la cadena de bloques}{buterin2014ethereum}
  
 \framet{Ethereum addresses}{
  Ethereum addresses behave similar to Bitcoin's.
\begin{block}{Example Ethereum user address}
0XBB49CD914B515B2050080440B9B7EE86DBFD22CC
\end{block}
These addresses can identify:
\begin{itemize}
    \item A user.
    \item A deployed smart contract.
\end{itemize}
}

\frameitcp{IPFS}{IPFS.png}{IPFS is a distributed file system that works over a P2P network}{benet2014ipfs}{0.7}

 \framet{IPFS addresses}{
  IPFS creates an address to access the file through its data hash:
\begin{block}{Example IPFS file address}
/ipfs/\textbf{QmS4ustL54uo8FzR9455qaxZwuMiUhyvMcX9Ba8nUH4uVv}
\end{block}
These addresses identify uniquely each file in the system.
}

\section{Plataform description}

\subsection{Distributed design}

\frameit{Distributed design}{eipfs.png}{The platform uses \textbf{IPFS} as file system and \textbf{Ethereum} as management system to controll the publication process}

\framet{Distributed design}{The platform is able to control the publication process through the following interactions:
\begin{itemize}
    \item Paper submission.
    \item Reviewer proposal.
    \item Review task acceptance.
    \item Review submission.
    \item Review rating.
\end{itemize}}

\subsection{Main features}
\framei{Open Access By design}{eipfs.png}

\framet{Reviewer reputation system}{The reputation system for reviewers allow the following interactions:
\begin{itemize}
    \item All reviews are public and accessible.
    \item Each review can be rated by the authors, the editors of the journal and the other reviewers.
    \item Reviewers get rewarded for doing a \emph{good job} and get penalized otherwise.
    \item Reviews can have a delivery time.
\end{itemize}
}

\framet{Transparent governance}{The proposed system aims to improve the peer review process transparency, speed,
and fairness.
\vspace{10mm} \\
With the interactions being time-stamped and
tamper-proof thanks to the blockchain technology, they can be monitored,
audited, and held accountable.
\vspace{10mm} \\
This transparency, combined with a distributed infrastructure for peer review,
facilitates the exploration of new workflows.
}

\section{Architecture}
\subsection{Platform design}
\framei{Platform architecture}{architecture.png}
\subsection{First Prototype}
\framei{First Prototype}{UML_class_1.png}
\subsection{Second Prototype}
\framei{Second Prototype}{UML_class_2.png}

\section{Proof of concept}
\subsection{Platform font-end}
\framei{Proof of concept}{mockup.png}
\framei{Proof of concept}{homepage.png}
\framei{Proof of concept}{journal.PNG}
\framei{Proof of concept}{paper.PNG}
\framei{Proof of concept}{rating.png}

\section{Discussion}
\subsection{Issues and possible solutions}
\framei{Issues and possible solutions}{privacyReviewRating.jpg}

\subsection{Implications}
\framet{Main implications}{
\textbf{Creating} a distributed publication system for open science could impy:
\begin{itemize}
  \item A \textbf{reduction} of costs in money and time of the whole process both for authors and readers.
  \item The exploration of \textbf{altenative models} for science publishing, impact factors and citation standards.
  \item \textbf{Prevention} of possible discrimination problems due to gender or rivalry research causes.
\end{itemize}
}

\section{Conclusion}
\subsection{Concluding remarks and Future work}
\framet{Concluding remarks}{
\begin{itemize}
  \item This work proposes the design and development of a decentralized platform for open science.
  \item Besides the fact that this platform is only a prototype, it has a lot of potential and opens new lines of reseach.
  \item The proposed system's infrastructure relies on new technologies with their own challenges and problems.
  \item Other open issues that require further research and may be explored in future work.
\end{itemize}
}
\framet{Future work}{
\begin{itemize}
  \item Solve privacy problems with the system proposed in the discussion chapter.
  \item Test the prototype with real users.
  \item Improve the platform publication process to automate steps like reviewer proposal.
  \item Create a larger project to make a future PhD.
\end{itemize}
}

\framet{Conferences}{
\begin{itemize}
\item \textbf{EthCC2018} - Ethereum community conference. \emph{Paris, France} (March
  2018). \url{https://ethcc.io/}

\item \textbf{PEERE2018} - International Conference on Peer Review. \emph{Rome, Italy}
  (March 2018). \url{http://www.peere.org/conference/}

\item \textbf{SPONBC2018} - Scientific Publishing on the Blockchain. \emph{Vienna,
    Austria} (May 2018).
  \url{https://blockchainforscience.com/2018/02/09/sponbc2018/}
\end{itemize}

}

\framet{Papers published}{
Tenorio-Fornés, A., Jacynycz, V., Llop, D., Sánchez-Ruiz, A.A., Hassan, S.
\textbf{``A Decentralized Publication System for Open Science using Blockchain
  and IPFS''}. \emph{PEERE International Conference on Peer Review Proceedings.}
Rome (2018). Status: Published.
\vspace{10mm} \\
Tenorio-Fornés, A., Jacynycz, V., Llop, D., Sánchez-Ruiz, A.A., Hassan, S.
\textbf{``Towards a Decentralized Process for Scientific Publication and Peer
  Review using Blockchain and IPFS''}. \emph{HICSS Hawaii International Conference
  on System Sciences (CORE-A)}. Hawaii (2019). Status: Accepted.
  }

\section{References}
\begin{frame}[allowframebreaks]{References}
\bibliography{references}
\printbibliography
\end{frame}

\end{document} 