

\documentclass{beamer}

\usetheme{Madrid} 
\setbeamertemplate{section in toc}{\inserttocsectionnumber.~\inserttocsection}
\setbeamertemplate{subsection in toc}{
  \hspace{5mm}\inserttocsectionnumber.\inserttocsubsectionnumber.~\inserttocsubsection
  \breakhere}
\setbeamertemplate{subsubsection in toc}{}
\setbeamertemplate{navigation symbols}{} % To remove the navigation symbols from the bottom of all slides uncomment this line


\AtBeginSubsubsection[]{
  \begin{frame}
   \frametitle{Parte \insertsectionnumber: \insertsectionhead}
    
    \begin{center}
      \huge{\insertsubsectionhead}
    \end{center}
    \vspace{5mm}
    \begin{center}
      \begin{beamercolorbox}[sep=8pt,center,shadow=true]{title}
        \usebeamerfont{title}
        \insertsubsubsectionhead\par%\\
      \end{beamercolorbox}
    \end{center}
  \end{frame}
}

\newcommand{\ib}[1]{\item \textbf{#1} }

\usepackage{graphicx} % Allows including images
\usepackage{booktabs} % Allows the use of \toprule, \midrule and \bottomrule in tables
\usepackage{import}
\usepackage{eurosym}
\usepackage[utf8]{inputenc}
\usepackage[spanish]{babel}
\usepackage{media9}    
\usepackage{etoolbox} 
\usepackage{tikz}
\usepackage{subcaption}

\usepackage[sort&compress,sectionbib]{natbib}
\bibliographystyle{IEEEtranN}

\newcommand{\framei}[4]{
  \begin{frame}{#1}
    \begin{center}
      \includegraphics[width=0.9\linewidth,height=0.7
      \textheight,keepaspectratio]{img/#2}\\
      #3~\cite{#4}.
    \end{center}
  \end{frame}
}

\newcommand{\framein}[3]{
  \begin{frame}{#1}
    \begin{center}
      \includegraphics[width=0.9\linewidth,height=0.7
      \textheight,keepaspectratio]{img/#2}\\
      #3.
    \end{center}
  \end{frame}
}

\newcommand{\framet}[3]{
  \begin{frame}{#1}
    #2
  \end{frame}
}
\title[Descentralizando la ciencia]{\huge{Descentralizando la ciencia:}\\
  Plataforma distribuida para publicación de artículos académicos}

\author[V. Jacynycz]{Viktor Jacynycz Garc\'ia}

\institute[UCM] % Your institution as it will appear on the bottom of every slide, may be shorthand to save space
{
  Grupo Grasia\\
  Facultad de informática\\
  Universidad Complutense de Madrid \\ % Your institution for the title page
  \medskip
  \textit{vsjg@ucm.es} % Your email address
}
\date{} % Date, can be changed to a custom date

\setcitestyle{numbers,square}
% ~\citestyle{nature}
\bibliographystyle{unsrtnat}

\begin{document}
\begin{frame}
\titlepage % Print the title page as the first slide
\end{frame}

\begin{frame}
\frametitle{\'Indice}
\tableofcontents 
\end{frame}

\section{Contexto}
\subsection{Contexto Socio-cultural}
\subsubsection{Sistemas de publicación}
\framei{Sistemas de publicación}{journal.png}{Los primeros \emph{journals} comenzaron a
  surgir a partir de 1665}{kronick1976history}

\framei{Método de publicación}{publishing.png}{Antiguamente las ediciones de los
  \emph{journals} se imprimían en papel a través de las imprentas}{spier2002history}

\framei{El proceso de revisión por pares (\emph{1972})}{peerreview.png}{El proceso de revisión
  por pares determina la eligibilidad de un artículo para ser publicado en un \emph{journal}}{spier2002history}

\framei{Papel de los editores hoy en día}{publishers2.png}{Con la llegada de
  internet, el papel de los editores se cuestiona, ya que no es necesario
  imprimir los artículos para publicarlos}{lariviere2015oligopoly}

\subsubsection{Sistemas de reputación}
\framei{Sistemas de reputación}{reputation.png}{Los sistemas de reputación nos permiten
  confiar en terceros dentro de un sistema, sin tener que conocerlos
  previamente}{resnick2000reputation}

\framein{Sistemas de reputación}{reputation2.png}{Hoy en día muchos servicios
  utilizan estos sistemas}

\subsection{Contexto Técnico}
\subsubsection{Arquitecturas distribuidas}
\framei{Arquitecturas de las redes}{architectures.png}{Diferentes arquitecturas de una red}{baran1964distributed}

\framei{Blockchain}{blockchain.png}{Funcionamiento de la primera cadena de bloques}{nakamoto2008bitcoin}

\framei{Sistema basado en transacciones}{transaction.png}{Las transacciones generan un cambio de estado}{antonopoulos2014mastering}

\framei{Smart contracts}{sc.png}{Los contratos inteligentes son
  pequeños fragmentos de código que cambian el estado de la cadena de bloques}{buterin2014ethereum}

\subsection{Estado del arte}
\subsubsection{Métodos alternativos de publicación}
\framet{Métodos alternativos de publicación}{
  \begin{itemize}
    \ib{\emph{Open Journal Systems}:} Programas diseñados para facilitar el
    proceso de publicación para \textit{journals} open access~\cite{willinsky2005open}.
    \ib{Preprints:} Son artículos pendientes de publicación accesibles
    para cualquier persona~\cite{shuai2012scientific}.
    \ib{Mega-journals:} Combinación de varios \emph{journals} en uno solo para
    fomentar la idea del open access~\cite{binfield2013open}
    \ib{Publicación continua:} Metodología que utilizan algunos
      \emph{journals} online. Consiste en publicar directamente los artículos
      aceptados~\cite{anderton2013continuous}.
  \end{itemize}
}

\subsubsection{Sistemas actuales de reputación}
\framein{Stack exchange como referencia}{se.png}{Es una de las comunidades más
  conocidas que implementa un sistema de reputación}

\framein{Sistema de rating}{amazon.jpg}{Amazon, Google y muchas otras empresas  tienen un sistema de
  puntuación basado en 5 estrellas}
\subsection{Objetivos}

\section{Plataforma}
\subsection{Descripción general}
\subsubsection{Funcionamiento}
\subsubsection{Arquitectura}
\begin{frame}
\frametitle{Conclusiones}
\end{frame}

\section{Referencias}
\begin{frame}[allowframebreaks]{Referencias}
\bibliography{references}

\printbibliography
\end{frame}

\end{document} 