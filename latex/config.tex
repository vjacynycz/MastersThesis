%---------------------------------------------------------------------
%
%                          config.tex
%
%---------------------------------------------------------------------
%
% Contiene la  definici_n de constantes  que determinan el modo  en el
% que se compilar_ el documento.
%
%---------------------------------------------------------------------
%
% En concreto, podemos  indicar si queremos "modo release",  en el que
% no  aparecer_n  los  comentarios  (creados  mediante  \com{Texto}  o
% \comp{Texto}) ni los "por  hacer" (creados mediante \todo{Texto}), y
% s_ aparecer_n los _ndices. El modo "debug" (o mejor dicho en modo no
% "release" muestra los _ndices  (construirlos lleva tiempo y son poco
% _tiles  salvo  para   la  versi_n  final),  pero  s_   el  resto  de
% anotaciones.
%
% Si se compila con LaTeX (no  con pdflatex) en modo Debug, tambi_n se
% muestran en una esquina de cada p_gina las entradas (en el _ndice de
% palabras) que referencian  a dicha p_gina (consulta TeXiS_pream.tex,
% en la parte referente a show).
%
% El soporte para  el _ndice de palabras en  TeXiS es embrionario, por
% lo  que no  asumas que  esto funcionar_  correctamente.  Consulta la
% documentaci_n al respecto en TeXiS_pream.tex.
%
%
% Tambi_n  aqu_ configuramos  si queremos  o  no que  se incluyan  los
% acr_nimos  en el  documento final  en la  versi_n release.  Para eso
% define (o no) la constante \acronimosEnRelease.
%
% Utilizando \compilaCapitulo{nombre}  podemos tambi_n especificar qu_
% cap_tulo(s) queremos que se compilen. Si no se pone nada, se compila
% el documento  completo.  Si se pone, por  ejemplo, 01Introduccion se
% compilar_ _nicamente el fichero Capitulos/01Introduccion.tex
%
% Para compilar varios  cap_tulos, se separan sus nombres  con comas y
% no se ponen espacios de separaci_n.
%
% En realidad  la macro \compilaCapitulo  est_ definida en  el fichero
% principal tesis.tex.
%
%---------------------------------------------------------------------


% Comentar la l_nea si no se compila en modo release.
% TeXiS har_ el resto.
% ___Si cambias esto, haz un make clean antes de recompilar!!!
\def\release{1}


% Descomentar la linea si se quieren incluir los
% acr_nimos en modo release (en modo debug
% no se incluir_n nunca).
% ___Si cambias esto, haz un make clean antes de recompilar!!!
%\def\acronimosEnRelease{1}


% Descomentar la l_nea para establecer el cap_tulo que queremos
% compilar

% \compilaCapitulo{01Introduccion}
% \compilaCapitulo{02EstructuraYGeneracion}
% \compilaCapitulo{03Edicion}
% \compilaCapitulo{04Imagenes}
% \compilaCapitulo{05Bibliografia}
% \compilaCapitulo{06Makefile}

% \compilaApendice{01AsiSeHizo}

% Variable local para emacs, para  que encuentre el fichero maestro de
% compilaci_n y funcionen mejor algunas teclas r_pidas de AucTeX
%%%
%%% Local Variables:
%%% mode: latex
%%% TeX-master: "./Tesis.tex"
%%% End:
