\chapter{Introduction}
\label{intro}
\begin{FraseCelebre}
  \begin{Frase}
    Who Watches the Watchmen?
  \end{Frase}
  \begin{Fuente}
    Watchmen - Alan Moore
  \end{Fuente}
\end{FraseCelebre}

Scientific research nowadays is based on publishing in journals with a high
journal impact factor (\emph{JIF})\cite{doi:10.1001/jama.295.1.90}, a
researcher's career can be measured depending on the number of papers published
in these journals. One of the most well-known is the Journal Citation Reports
(\emph{JCR}), an indicator that represents the relation between the number of
citable items and the number of citations they get. This impact factor is also
divided in four quartiles that determine the raking of a certain journal,
meaning that a journal in the first quartile (Q1) has higher impact factor than
one in the second, third or fourth
quartiles\footnote{https://jcr.incites.thomsonreuters.com/}. \emph{JCR} was
originally an evolution of the Science Citation Index, born in 1955
\cite{garfield2007evolution} and nowadays managed by a private company called
Thomson Reuters\footnote{https://www.thomsonreuters.com}.

Journals with high \emph{JIF} want to maintain their impact factor have to make
sure that the papers that come out in their issues have a large number of
citations, so commonly they are going to look for novel and high-impact ones.
Therefore, the editors of these journals will have a pool of reviewers on which
they can trust to review these papers. But sometimes these reviews are not
entirely objective, since there are cases of unfavorable reviews due to gender
causes, especially in scientific fields \cite{wenneras2001nepotism}.

Besides, it is necessary to consider that the time of revision for an paper is
excessively long, causing the process of academic investigation to be quite slow
\cite{huisman2017duration}.

The second problem is that the benefits of scientific distribution are
centralized in publication systems, nor the authors, the reviewers or the
readers get money from it. Today, with electronic paper distribution,
universities purchase site licenses for online access to journal contents. This
system implies an additional cost for the universities who want to advance in
their research fields and do not have enough money for it. However, site
licenses are not always disadvantageous. Some journals issued by private
companies and universities adjust their prices to maximize subscriptions
\cite{bergstrom2004costs}. But generally, people who earn money from this
paper-based system only act as an intermediary between the authors, the
reviewers and the readers.

The internet offers the possibility to meet people all around the world, and
when it comes to trust total strangers, you should have a system in which you
can rely on to deposit your trust in them. Reputation systems are the solution
to these problems, since they offer a good first impression about an unknown
person \cite{resnick2000reputation}.

Editors who want to assign the review of a paper to a series of reviewers have
to rely on them beforehand. Thus, limiting the spectrum of fields that can be
revised to the fields in which those reviewers are experts. If you want to
broaden the scope of reviewers with more fields of expertise, you need to
contact new reviewers. But there is no easy way to predict reviewer quality from
their training and experience factors \cite{callaham_relationship_2007}, so a
rating system of reviewers would be useful for journals to select the best
reviewers. The solution is a reviewer reputation network, in which reviewers get
rated based on their reviews and build up their reputation based on good
practices and helpful reviews. In this network, publishers who have to find new
reviewers for their papers do not have to know them beforehand, since trust is
placed in the reputation network instead of in the person itself.


Science publication and peer review are build on a paper-based paradigm, with
only a few changes in the last centuries~\cite{spier2002history}. Critics to
current science publication and peer review systems include concerns about its
fairness~\cite{wenneras2001nepotism}, quality~\cite{goldbeck1999evidence},
performance~\cite{huisman2017duration}, cost~\cite{bergstrom2004costs}, and
accuracy of its evaluation processes~\cite{doi:10.1001/jama.295.1.90}, among
others.

The development of the Internet enabled the proposal of alternatives for science
dissemination~\cite{eysenbach2006citation} and
evaluation~\cite{walker_emerging_2015}. The reduction of distribution costs
enabled a wider access to scientific knowledge, and questioned the role of
traditional publishers~\cite{ReinventingRigor}. It is acknowledged that the Open
Access and Open Science movements have successfully reduced the economic cost of
accessing knowledge to readers~\cite{evans2009open}. However, it has not
successfully challenged traditional publishers' business
models~\cite{lariviere2015oligopoly}, who are now combining charging readers and
charging authors~\cite{van2013true}.

Peer review has suffered multiple criticism, and yet only marginal alternatives
have gathered success~\cite{ware2008peer}. The literature provides multiple
proposals around open peer review~\cite{ford2013defining}, and proposals of
reputation networks for reviewers~\cite{frishauf2009reputation}. In fact, a
start-up, Publons\footnote{https://publons.com/}, provides a p latform to
acknowledge reviews and open them up.

\section{Objective}

We aim to challenge middlemen such as traditional publishers in science
publication. Particularly, we propose a decentralized publication system for
open science, allowing 1) paper submissions, 2) assignment of reviewers, 3) peer
review and, as a novelty, 4) the rating of peer reviews. With this distributed
system, we aim to improve the quality and efficiency of reviews and knowledge
distribution, helping editors, authors, and reviewers:
\begin{itemize}
\item Editors and journals will be able to find the best peer reviewers in their
  fields of interest, and also those that respond quickly. Thus, reducing
  time-to-publish and publishing costs.
\item Authors will be able to submit papers to time-responsive, free, open
  access journals, and forget about slow, unfair and unaccountable anonymous
  reviews.
\item Reviewers will finally have their work recognized.
\end{itemize}

We are interested in exploring the following challenges, that could be dealt
with our technology:
\begin{itemize}
\item Reduce time-to-review by rewarding on-time reviewers.
\item Measure and prevent sexism, nepotism and other abuses in peer review.
\item Develop fully autonomous decentralized journals.
\item Explore fully free publication systems for Open Access science, while
  enabling innovative business models.
\item Explore alternative and open metrics for papers, journals and reviewers.
\end{itemize}

  \section{In this work}
  In this work there will be the following sections:

  \subsubsection*{Part 1: Preface.}
  \begin{itemize}
    \itbf{Background and State of the art:} Background about the scope of the
    project and what technologies are trying to change the current publication
    systems and how is affecting the scientific community.

    \itbf{Methodology and Technology:}Methodology followed during the
    realization of this work, and the technologies used to implement the
    platform's architecture.
  \end{itemize}
  \subsubsection*{Part 2: Decentralized Science.}
  \begin{itemize}
    \itbf{Platform description:} Platform general description, featuring its
    main strengths regarding the current platforms and explaining how it works
    and what is the expected behavior if it is widely used in the future.

    \itbf{Architecture:} Technical description of the platform, including the
    front end architecture and the definition of the smart contracts'
    infrastructure, and the process followed to reduce the interaction costs.

    \itbf{Product:} Proof of concept of the platform, how it works and how the
    users can interact with the blockchain and with the p2p file network.

    \itbf{Discussion:} Results obtained after the realization of the work
    proposed in this project, how it will affect the scientific community and
    how to measure the potential impact of the platform.

    \itbf{Conclusion and future work:} Implications of this work in the
    scientific community and the next steps to follow to create an ecosystem of
    autonomous publication systems, without the need of middlemen such as
    journals or editors, and a proposal of a future Ph.D. about this subject.
  \end{itemize}


%%%
%%% Local Variables:
%%% mode: latex
%%% TeX-master: "../Tesis.tex"
%%% End:
