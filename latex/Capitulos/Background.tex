\chapter{Background and State of the art}

\begin{FraseCelebre}
  \begin{Frase}
    The future is already here – it's just not evenly distributed.
  \end{Frase}
  \begin{Fuente}
    William Gibson
  \end{Fuente}
\end{FraseCelebre}

% -------------------------------------------------------------------
% \section{Cooler Section}
% -------------------------------------------------------------------

\section{Socio-cultural background}
\label{scb}
\subsection{Publication systems}
\label{scb:ps}
The methodology of scientific publications creation was established in
1620~\cite{bacon1878novum}, in which Francis Bacon established certain steps to
elaborate what we know today as scientific papers. But it was 45 years later, in
1665, when appeared the considered the first scientific journal:
\emph{Philosophical Transactions of the Royal
  Society}~\cite{kronick1976history}. At that time, editors were the ones who
had to carry out the revision of the papers that would be published in these
journals.

% El proceso de generación de publicaciones científicas actual se estableció en
% 1620~\cite{bacon1878novum} https://pbs.twimg.com/media/DY0rMNmU0AAoVQz.jpgy
% fue en 1665 cuando comenzaron a aparecer lo que hoy consideramos como los
% primeros journals científicos~\cite{kronick1976history}. En esa época los
% editores eran los que tenían que llevar a cabo la revisión de los artículos
% que serían publicados en dichos journals.

It was around 100 years later when an alternative system was adopted, instead of
charging the editors with the work of reviewing all the papers, a group of
experts decide if each one is good enough to be published or not. This is the
beginning of process known today as ``peer review''~\cite{spier2002history}.

% No es hasta unos 100 años después cuando se empieza a adoptar un nuevo sistema
% de revisión: todos los papers enviados a la revista serían revisados por
% expertos que juzagarían si debe ser publicado o no~\cite{spier2002history}.

But the process of scientific publication as we know it today was settled down
in the 19th century, with the actual peer review
process~\cite{burnham1990evolution}, stablishing the guidelines of the
paper-based paradigm that we have in science nowadays.

% Pero el proceso de publicación científica de hoy en día se asento en el siglo
% 19, junto con el sistema actual de peer review
% editorial~\cite{burnham1990evolution}, estableciendo las pautas del
% paper-based paradigm por el que se rige el proceso de investigación científica
% acutal.

Determining a the quality of a scientific paper is difficult, but today we have
different ways to do so. To be able to estimate this quality, usually there are
two approaches: before and after publishing.

\textbf{Peer review process} consists on a group of researchers in a certain
field that decide a paper's eligibility to be published. They read the paper,
and submit a review and an ``acceptance score'' representing if they think the
paper should be accepted, have a revision or be directly rejected. Normally
these researchers are unknown to the authors and the reviews are made
anonymously (\emph{blind review}). Sometimes reviewers also do not know who the
authors are (\emph{double blind review}). And in rare cases, both the authors
and the reviewers are public (\emph{open review}).This process can determine the
quality of a paper before it is published~\cite{szklo2006quality}.

% Hay dos formas de determinar la calidad de un paper: antes de ser publicado, y
% despues de ser publicado. El proceso de peer review es el que determina la
% calidad de un paper antes de ser publicado, ya que un grupo de expertos en la
% materia juzgaran si el paper es de calidad o no~\cite{szklo2006quality}. Este
% proceso puede suponer un indicador de la calidad de un paper si los revisores
% son buenos, pero como se mencionó en la sección \ref{intro}, es dificil
% determinar si un investigador académico es buen revisor o
% no~\cite{callaham_relationship_2007}.

% Another way to determine this quality is before it is published, obtaining the
% number of citations it gets over time~\cite{redner1998popular}. Normally a
% paper with a higher number of citations is considered to have better quality
% than a paper with lower number of citations.

La otra forma de determinar la calidad de un paper es tras la publicación,
obteniendo el número de citas que tiene a lo largo del
tiempo~\cite{redner1998popular}. Esta forma de analizar la calidad da pie a la
generación de índices que determinan lo bueno que es investigador o no. Como por
ejemplo, el índice H que asigna un valor numérico a cada investigador en función
de los artículos que ha escrito y las citas que ha
recibido~\cite{bornmann2007we}. Respecto a estas métricas se han explorado
varias alternativas, ya que determinar si un investigador es bueno o no a través
de un valor no siempre es acertado~\cite{bornmann2008there}. Sin embargo, en la
mayoría de los casos, es posible predecir a través de éste índice si la calidad
de los futuros paper de un investigador tendrán buena calidad o
no~\cite{hirsch2007does}.

Si el índice H es uno de los factores más usados para ver el impacto que tienen
los investigadores, el factor más utilizado y reconocido que determina el
impacto de un journal científico es el JCR~\cite{doi:10.1001/jama.295.1.90}
(comentado en la sección~\ref{intro}).

El sistema de investigación academica actual se basa en la publicación de
artículos en journals y conferencias, ya que los artículos que forman parte de
una investigación que no son publicados en uno de estos sitios son llamados
``grey literature''~\cite{rothstein2009grey}.

Las personas que buscan hacer carrera en el mundo académico, una de las
prioridades es publicar el máximo número de artículos en journals de alto
impacto. En paises como España, entidades como la
ANECA~\footnote{www.aneca.es/}, determinan si una persona puede dar clase en la
universidad basado en una serie de factores. Uno de los más importantes es el
número de pubicaciones con JRC alto que tenga dicha persona.

Las editoriales (publisers) eran las encargadas de producir, imprimir y
distribuir las ediciones de los \emph{Journals} cuando estos empezaron a surgir.
Muchos publisers importantes como
Willey-Blackwell~\footnote{https://www.wiley.com},
Elsevier~\footnote{https://www.elsevier.com/} o
Springer~\footnote{http://www.springer.com/} siguen existiendo desde principios
del siglo 19. Pero la era digital ha provocado que copiar un documento, que
antes era un proceso costoso, ahora tenga un coste ínfimo. Aún asi, los
publishers se siguen lucrando de este sistema, actuando de intermediarios entre
las personas que crean ciencia y las que la
consumen~\cite{lariviere2015oligopoly}. En la era en la que replicar la
información no supone un coste, el papel de los publishers debe ser cuestionado
y se debe migar a sistemas de publicación y difusión científica más justos y
honestos.

En el sistema acutual de publicación, gran parte del trabajo lo realizan los
revisores que permanecen anónimos tanto para los autores como para el resto de
la comunidad científica. Este trabajo propone, entre otras cosas, realizar un
sistema de reputación de revisores para que estos puedan obtener el crédito y
reconocimiento de realizar buenas revisones, y sean penalizados cuando realicen
malas.

\subsection{Reputation systems}
\label{scb:rs}
Un sistema de reputación es una tecnología que permite a los usuarios que la
utilizan confiar en terceros que también utilicen el mismo sistemas, sin que
estos se conozcan. El funcionamiento de un sistema de reputación lo que hace es
recolectar, agregar y distribuir los comentarios recibidos del comportamiento
anterior de los participantes. Esta idea nació a principios de los años 2000 en
el que el uso de internet se extendió por todo el mundo, y eran necesarios
sitemas para poder confiar en desconocidos~\cite{resnick2000reputation}.

% Reputation systems today arise from the need to trust unknown
% individuals~\cite{resnick2000reputation}.

La ``reputación'' es normalmente un valor que indica cúanta confianza tiene la
comunidad en un usuario. Esta reputación es ganada a través de interacciones con
el resto de las personas que utilizan el servicio o plataforma que implementa el
sistema de reputación. Algunas veces estas plataformas ofrecen
\emph{privilegios} a las personas que tengan cierto nivel de reputación,
desbloqueando ciertas acciones que sólo se pueden realizar llegado a cierto
umbral.

La idea básica de los sistemas de reputación es dar a los usuarios la
posibilidad de poder puntuar las interacciones que ocurren entre ellos. Una de
las plataformas pioneras en implementar este sistema fué
Ebay\footnote{https://ebay.com}, una web de compra-venta de segunda mano. En
ella, cada uno los vendedores tenía una reputación en función de si las
anteriores transacciones habían sido honestas o no. Cada usuario que comprara
productos de dicho vendedor tenía la oportunidad de puntuar la compra. De esta
manera, todos aquellos vendedores que intentaban timar a los usuarios,
inmediatamente recibían mala reputación.

En este trabajo se consideran dos tipos de sistemas de puntuación: el primero es
un sistema de puntiación sobre 5 estrellas, en el que los usuarios puntúan del 1
al 5 a los otros usuarios, siendo 1 mala reputación y 5 buena
reputación~\cite{}. Este sistema de reputación se utiliza en en muchas webs de
comercio electrónico. El segundo es el like/dislike, en el cual las
interacciones se evaluan con valoración positiva o valoración negativa, en
función de la interacción que hayan tenido dos usuarios~\cite{}.

Este sistema se ha extendido a muchos servicios de internet en los que no existe
una entidad centralizada en la que los usuarios tengan que confiar, por lo que
construir un sistema todalmente distribudo requiere una confianza entre los
usuarios que puede ofrecer un buen sistema de reputación. En la sección
\ref{soa:rs} se exploran diferentes sistemas que existen acutalmente y formas de
mejorarlos.

\section{Technical background}
\label{tb}
\subsection{Network Architectures}
\label{tb:na}
\figura{architectures.png}{width=0.95\linewidth}{tb:na:diagram}{Digram of the
  diferent network architectures} Una de los conceptos más importantes cuando se
habla de Ethereum es el de arquitectura distribuida. Propone eliminar la
centralización quitando la dependencia de un servidor centralizado.

Existen tres tipos importantes de estructuras (ver figura 3):

Estructura Centralizada: Toda la estructura está gestionada por un solo nodo y
sus usuarios pertenecen a la misma comunidad. Se utiliza principalmente en
servicios web, alojadas en un servidor centralizado por el que tienen que pasar
todas las personas que quieran acceder a ella (e.g. Wikipedia, Airbnb, Github).

Estructura Federada: La infraestructura está dividida en varios nodos operativos
que funcionan como su propia estructura centralizada, fragmentado el servidor
central en pequeños servidores distribuidos. Cada uno de estos tiene su propio
dueño y su comunidad. A pesar de esto, cualquier usuario de la estructura puede
acceder a los datos de otros independientemente del nodo al que pertenezcan
(e.g. GNU Social, Buddycloud, Diaspora).

Estructura P2P (Peer to Peer): Es una estructura totalmente distribuida,
particionando los trabajos y la información entre los usuarios de la red
llamados “Peers”. Cada uno de estos tienen los mismos privilegios en la
infraestructura. Cada usuario controla su aportación a la red distribuida y
generalmente todos pertenecen a la misma comunidad (e.g. BitTorrent, Twister,
Bitcoin, Ethereum).


Uno de los ejemplos más importantes de una arquitectura distribuida P2P es el de
Bitcoin, en el que todos los usuarios ven todas las transacciones y tienen los
mismos derechos.

Ethereum utiliza también la arquitectura P2P al igual que Bitcoin consiguiendo
así que todo el software desarrollado en este lenguaje sea 100\% libre y
descentralizado, ya que todos los usuarios de la red tienen acceso a todos los
contratos de la cadena de bloques y al código fuente de estos.
\subsection{Criptocurrencies}
\label{tb:cryptos}
En la década de los 80 y los 90 empezaron a aparecer formas de pago
descentralizadas como ecash~\cite{chaum1995introduction} que ofrecían una moneda
con un alto nivel de privacidad; fue entonces cuando empezó a surgir el concepto
del “dinero electrónico anónimo”.

Uno de los pioneros en la creación de una moneda electrónica descentralizada fue
Wei Dai, que en 1998 publicó su propuesta de dinero electrónico llamada
B-money~\cite{dai1998b}. Wei Dai proponía la creación de dinero mediante la
resolución de puzles computacionales. A partir de la idea de Wei Dai han ido
surgiendo otras propuestas como Bit gold~\cite{szabo2008bit} para mejorar la
implementación de una criptomoneda haciendo uso de RPOW~\cite{finney2005rpow},
una extensión del sistema de prueba de trabajo de Hashcash~\cite{back1997hash}.

No fue hasta el año 2009 que se implementó por primera vez una moneda
descentralizada, cuando Satoshi Nakamoto publicó la primera versión de
Bitcoin~\cite{nakamoto2008bitcoin}. Esta moneda tenía como objetivo crear un
sistema de pagos electrónicos completamente descentralizado, empleando pruebas
criptográficas en lugar de confianza mediante un sistema de proof of work o
prueba de trabajo. Este mecanismo criptográfico resolvía dos problemas. Primero,
ofrecía una forma efectiva de alcanzar un consenso entre todos los nodos de la
red sobre el estado de la misma. Segundo, ofrecía la posibilidad de que
cualquier persona podía unirse a la red de nodos, resolviendo los problemas
políticos que implican el hecho de quién llevaba el control de la red de nodos.


\subsection{Blockchain}
\label{tb:blockchain}
\lip

\subsection{Smart Contracts}
\label{tb:smartcontracts}
\lip

\section{State of the art}
\subsection{Alternative publication systems}
\label{soa:aps}
Publication systems, as seen on section \ref{intro} are vampirizing the
industry. However, there are some attempts to change this paradigm on behalf of
science dissemination.

\bb{Open journal systems}~\cite{willinsky2005open} is an open software designed
to facilitate the publishing process. This project was created by the Public
Knowledge Project\footnote{https://pkp.sfu.ca/about/} and it targets open-access
online journals that want to speed up the publication processes. The system
provides tools to control the whole publishing process from paper submission,
through peer reviewing to the final publication issue.

\bb{Mega-journals}~(or Multi-journals)~\cite{binfield2013open,wellen2013open}
combine multiple journals into a single journal, allowing the publication of
open-access papers, which have gone through a peer review process. The first
journal to adopt this idea was the \emph{PLOS ONE}
Journal\footnote{http://journals.plos.org/plosone/} as of the project
\emph{Public Library of Science}. This project aims to create a library of
scientific journals under the values of open access and creative commons
licenses. As a result of the success of the \emph{PLOS ONE} journal, other
publishers have started their own mega-journals. Featuring alternative impact
metrics, reusability of figures and data, post-publication discussions and
portable reviews from other journals~\cite{bjork2015have}.

The \bb{continuous publication} model is based on publishing individual papers
migrating from the previous issue-based model~\cite{anderton2013continuous}.
This method is seen as an altenative for open-access journals as it speeds up
the publication process~\cite{haymanview}. \emph{DPSOS}\footnote{Decentralized
  Publication System for Open Science} adopts this model by design (see
section~\ref{tech:sec:ethereum:sm}) as it publish automatically papers that meet
certain preconditions that are written in the blockchain.

\bb{Preprints} are scientific papers that have not yet gone through the peer
review process~\cite{harnad2003electronic}. Formerly, the preprints that were
sent to the journals were private, and only accessible by the editors and
assigned reviewers. But nowadays it is common to publish a preprint before
sending it to a journal, uploading it to specialized platforms like
arXiv\footnote{https://arxiv.org/} or
Preprints\footnote{https://www.preprints.org/}~\cite{brown2001volution}. In fact
there is a correlation between the upload of a preprint and early citations
after the publication of the paper~\cite{shuai2012scientific}. This system is a
possible solution to the cold-start problem that papers of new researchers who
enter the academic career have~\cite{sugiyama2010scholarly}.

Social networks have also made a dent in the academic world, creating platforms
to contact other researchers and encouraging them to share open access papers.
Some of the well-known are Research Gate\footnote{https://www.researchgate.net},
Mendeley\footnote{https://www.mendeley.com} or
Academia\footnote{http://academia.edu}. But despite the good intentions of the
creators of these platforms, many of the journals demand the copyright of the
papers they publish, preventing the authors from sharing them through these
services.


Decentralized alternatives, in spite of their
promises~\cite{bartlingblockchain}, are still in their infancy. A few proposals,
none of them functional to date, have appeared recently.

One of them is a peer review proposal that tries to solve some of the peer
review socio-technical problems using cryptocurrencies~\cite{tennant2017multi}.
It needs a critical threshold of research community engagement, changing the
actual processes and platforms, to start being implemented.

Blockchain-enabled apps have also been proposed, with voting and storage of
publications. This is the case of Aletheia~\cite{morton2017aletheia}, a software
for getting open access papers published. This platform idea aims to use
blockchain as a decentralized and distributed database as a publishing platform.

Peer review quality control through blockchain-based cohort
trainings~\cite{dhillon2016bench} have been also proposed, with the promise of
transparency and decentralization using a distributed ledger. Research labs can
use this training network to test their technology and reduce the risk for
private investment opportunities.

Finally, some of the off-chain journals are adapting to the demands of the
current scientific community like Ledger\footnote{https://ledgerjournal.org}, a
cryptocurrencies and blockchain-based journal that records the publication
timestamps in the Bitcoin blockchain.

\subsection{Existing Reputation systems}
\label{soa:rs}
Many of the big internet communities like
Stackexchange\footnote{https://stackexchange.com/} or
reddit\footnote{https://www.reddit.com/} have their own reputation system.
Reputation systems behavior may vary depending on the
platform~\cite{josang2002beta}, but the most usual is the one where users get a
score based on certain interaction with the community.

Reputation systems also have a very large niche in e-commerce webs such as
Ebay\footnote{https://www.ebay.com}, in which people pay for a product sold by
an unknown vendor. There must be a previous trust in the vendor before buying
any product, so a reputation system offers a score given by other users that
encourages you to trust or not that certain seller~\cite{resnick2002trust}.

Reputation systems vary widely in scope, such as one for peer-to-peer
computing~\cite{zhou2007powertrust}, vehicle ad-hoc~\cite{dotzer2005vars}, web
services~\cite{moore2008reputation} and even Wikipedia~\cite{adler2007content}.
All of them are based on an exchange of trust between users who use these
services.

This same concept was intended to be transferred within the blokchain using a
token as a trust unit, which users exchanged as a sign of trust deposits among
them~\cite{sharples2016blockchain}.

But reputation systems also have problems when it comes to defend the users from
attacks to individuals~\cite{hoffman2009survey} and unfair ratings
~\cite{whitby2004filtering}, so the architecture chosen for it must consider
these weak points and try to mitigate them.

This paper proposes the development of a decentralized publication system for
open science. It aims to challenge the technical infrastructure that supports
the middlemen role of traditional publishers. Due to the successes of the Open
Access movement, some of the scientific knowledge is today freely provided by
the publishers. However, the content is still mostly served from their
infrastructure (i.e. servers, web platforms). This ownership of the
infrastructure gives them a position of power over the scientific community
which produces the contents~\cite{fuster2010governance}. Such central and
oligopolistic position in science dissemination allows them to impose policies
(e.g. copyright ownership, Open Access prices) and concentrate profits.

The proposed system aims to move the infrastructure control from the publishers
to the scientific community. It entails the decentralization of three essential
functions of science dissemination: 1) the peer review process, 2) the selection
and recognition of peer reviewers, and 3) the distribution of scientific
knowledge. The following section provides an overview of the system features,
while the final section discusses its challenges.
%%%
%%% Local Variables:
%%% mode: latex
%%% TeX-master: "../Tesis.tex"
%%% End:
