\chapter{Discussion}

\begin{FraseCelebre}
  \begin{Frase}
    Sooner or later you're going to realize just as I did that there's a
    difference between knowing the path and walking the path.
  \end{Frase}
  \begin{Fuente}
    Morpheus - The Matrix
  \end{Fuente}
\end{FraseCelebre}

\section{Issues and possible solutions}
\label{sec:privacyReview}

\figura{privacyReview.jpg}{width=0.7\linewidth}{PrivacyReviewRating}{Review
  privacy models}

Anonymity of reviewers and authors in peer reviews is traditionally used to
improve the fairness of the process. Thanks to single blind reviews, anonymous
reviewers can honestly critic a paper without fearing the reactions of the
authors. Double blind reviews also allow to reduce the impact of personal
biases. Finally, open review models propose that both authors and reviewers know
each other. These different privacy settings are shown in the Figure
\ref{PrivacyReviewRating}.But the anonymity of the reviewers can also be abused.
Unfair or low quality reviews were not discouraged by the system due to the lack
of consequences.

\subsection{Privacy for the peer review process}
\label{sec:privacy-peer-review}

As mentioned in previous sections, all the information is stored in either IPFS
(files) or Ethereum (interactions). All this information is public for everyone.
This may suppose a privacy problem for the reviewers if they do not want to
reveal their identity. Exploring different possible configurations for each model of
peer review may solve this problem.

Each of the anonymity options of the system requires different solutions, which
are discussed bellow. The question of whether we can keep the benefits of blind
review while providing accountability and recognition to reviews deserves
special consideration.

\subsubsection*{Blind peer review}
Blind review is the protection of the identity of reviewers in the peer review
process. This way, anonymous reviewers can honestly criticize a paper without
fearing the reactions of the authors. In a blockchain, this protection could be
easily achieved by using single-use addresses previously agreed with the editor.

\subsubsection*{Double blinded peer review}
A double blinded review is a blind review that additionally protects the authors
identity to prevent social bias~\cite{lee2013bias,budden2008double}. Authors
could protect their identities prior to publication by providing a single-use
public address on submission. Later they can reveal their real identity since
they are the only ones with access to that address.

\subsubsection*{Open peer review}

Open review models propose that both authors and reviewers know each
other~\cite{ford2013defining}. While studies found effect on the percentage of
reviewers declining to review~\cite{van1999effect} other implications remain
open to debate~\cite{groves2010open}.



% \subsection*{Blind peer review}
% Blind review is the protection of the identity of reviewers in the peer review
% process. In a blockchain, this protection could be easily achieved by using
% single use addresses or passwords agreed with the editor.

% \subsection*{Double blinded peer review}
% A double blinded review is a blind review that additionally protects the
% authors identity to prevent social
% bias~\cite{lee2013bias}~\cite{budden2008double}. Authors could protect their
% identities prior to publication by providing a single use public key from
% which later they sign their real identity or signing the paper with the hash
% of their names followed by a random constant, revealing the constant after
% acceptance.

% \subsection*{Open peer review}
% Open evaluation proposes the opening and deanonymization of peer
% review~\cite{ford2013defining}. While studies found effect on the percentage
% of reviewers declining to review~\cite{van1999effect} other implications
% remain open to debate~\cite{groves2010open}.

% Signed reviews are easy to implement by maintaining a public identity for the
% reviewer.

\subsection{Privacy for the rating process}
\label{sec:privacyRating}


\figura{privacyRating.jpg}{width=0.7\linewidth}{PrivacyRating}{Reviewer
  reputation privacy models}

Different privacy sections for peer review have been discussed in previous
subsection. Note, however, that the anonymity of the reviewers can be also
abused. Unfair and low quality reviews are not discouraged by the system due to
the lack of consequences. In order to alleviate this problem, our system
proposes the construction of a reputation network of peer reviewers so that
reviewers are awarded or criticized according to their work. This reputation
network can also adopt different privacy settings, allowing both anonymous and
signed ratings of either signed or anonymous reviews as depicted on Figure
\ref{PrivacyRating}. Next, we discuss the different privacy models for the
proposed reviewer reputation network.

% where review reports are rated by authors, reviewers and editors. The possible
% privacy settings of this system are presented in the following subsections.

\subsubsection*{Open Rating}
Similarly to open reviews, open ratings are easy to implement by maintaining a
public identity for the raters and reviewers.

\subsubsection*{Anonymous Rating}

Protecting the identity of raters is interesting in several reputation systems.
We can support this anonymity feature using \emph{blinded tokens}
~\cite{schaub2016trustless} that grant permission to rate without revealing the
identity of the rater. People authorized to rate a review, such as authors,
editors and other reviewers involved in the process, may each get one of these
tokens.

\subsubsection*{Rating Anonymous Reviews}

The question of whether we can keep the benefits of blind review while providing
accountability and recognition to reviewers deserves special consideration.

In a system that support voluntary signing of reviews, unsigned reviews would
not affect the reviewers reputation unless they acknowledge their authorship.
Thus, reviewers may only reveal their identities for well rated reviews,
reducing the desired accountability for poor quality, unfair or late reviews.

A system allowing anonymous, yet accountable, reputation system for peer
reviewing is therefore of great interest. Following, we discuss the feasibility
of adopting different anonymity and accountability approaches to realize this
novel system.

\subsubsection*{Accountability}

\emph{Collateral models} are widely used in blockchain technology to ensure that
an actor assumes negative consequences of an interaction in order to avoid the
greater consequence of loosing the collateral. A similar strategy can be used
for the anonymous reputation network. If a reputation collateral is requested
from the reviewers, they would be encouraged to claim even negative ratings.

\subsubsection*{Anonymity}

% \emph{Coin mixing} protocols are designed to obfuscate the relation between
% senders and receivers of Bitcoin payments by mixing in a single transaction
% many senders and receivers ~\cite{meiklejohn2015privacy}. We can not directly
% apply this approach to rate reviews since the receiver identity is known.
% However it can be used in collaboration with other techniques discussed below.

The previously discussed accountability tools can be combined with anonymity
measures to ensure accountable yet anonymous, peer reviews. Following we
introduce blockchain anonymity tools that could be used to implement this
system.\commsem{ (TODO: I feel like this section is too ethereal, and I think we
  have the risk to be rejected because of that.)}

\emph{Reusable payment codes} enable the possibility of using a large amount of
addresses to receive a payment ~\cite{harrigan2016unreasonable,
  ranvierReusable}. Reviewers may share one of these addresses with each of the
actors with permission to rate and then collect the reputation probably using an
anonymity layer such as coin mixing ~\cite{meiklejohn2015privacy}.
% Using a collateral model would encourage the acceptance of bad ratings.

\emph{ZK-Snarks} are a cryptographic tool enabling to prove a statement without
revealing anything else than the statement is in fact true (Zero-Knowledge Proof
of Knowledge) ~\cite{blum1988non,bitansky2013succinct}. They also provide this
property in a succinct and non-interactive fashion \comm{(i.e. using a
  relatively small proof and not requiring further communication between prover
  and verifier)}. Zcash uses this technology to build an anonymous
cryptocurrency~\cite{sasson2014zerocash}. A similar approach could be used to
manage anonymous ratings. With this technology, a reviewer could receive the
rating of a review she did without revealing from which review report or which
rating the reputation comes.
% As before, a collateral model can be used to encourage the acceptance of low
% ratings.


%%%
%%% Local Variables:
%%% mode: latex
%%% TeX-master: "../Tesis.tex"
%%% End:


Como resultado del desarrollo de esta plataforma, el escenario ideal sería que
algunos journals que ya hayan ido migrando al sistemas de publicación
alternativos como los explicados en la sección~\ref{soa:aps} se adapten a esta
plataforma. Al realizar un pequeño análisis del posible impacto de este trabajo
hay dos puntos importantes a destacar.

\section{Monetary Impact}
El impacto monetario sería uno de los más notables tras la implantación de este
sistema en sistemas de publicación científica de hoy en día. Según datos de
investigaciones al respecto, el coste de publicación de un artículo en una
revista de imacto varía de entre 1000\$ hasta los
5000\$~\cite{van2013true,russel2008business}, coste muchas veces inviable para
investigadores que quieran avanzar en la investigación científica.

El coste de la publicación y el acceso a la ciencia a través de el trabajo
propuesto sería únicamente variable en función del precio del ETH\footnote{La
  criptomoneda de Ethereum explicada en la sección \ref{tech:sec:ethereum}}.
Tras realizar un análisis con varias versiones de la plataforma desplegada se
determina que el coste de una transacción varía entre 100000 y 150000 de
gas\footnote{Gas es lo que pagas como comisión a la red de Ethereum por ejecutar
  una transacción}.

Teniendo en cuenta que para que se publique un paper han de realizarse como
mínimo 5 transacciones se puede determinar que el precio actual para publicar un
paper ronda entre los 4\$ y 6\$ segun datos de Ethereum Gas
Station\footnote{Precio de una transacción en Ethereum
  https://ethgasstation.info/}, más de 250 veces más barato que los sistemas de
publicación actual en el mejor de los casos.

\section{Review Time and Quality Impact}

Otro de los impactos importantes sería la reducción del tiempo y el aumento de
la calidad en el proceso de revision por pares.

Si se dispone de una masa crítica de usuarios de la plataforma propuesta, se
crea un ecosistema de usuarios que alimentan tanto la red de reputación de
revisores como los contratos de publicación científica (ver
section~\ref{contracts}). El proceso de Peer review se vería afectado de dos
maneras:

\begin{enumerate}
\item \textbf{El tiempo de revisión:} Los contratos inteligentes permiten
  establecer tiempos límites para la revisión de un artículo, suponiendo una
  penalización a los revisores que no cumplan estos plazos (ver sección
  \ref{reputation}). Si un Decentralized Journal tiene unos tiempos de revisión
  establecidos, y los revisores que asignan aceptan las revisiones,
  probablemente se experimente una mejoría en el tiempo de entrega de las
  revisiones y por lo tanto en el proceso de publicación, con respecto a los
  tiempos muchas veces excesivos de los sistemas de publicacion
  actual~\cite{huisman2017duration}.
\item \textbf{La calidad de las revisones:} Todas las revisiones son rateables
  por la comunidad y afectan directamente a la reputación del revisor, así que
  es altamente probable que la calidad de las revisiones en el sistema sufra una
  mejoría y desaparezcan muchos de los problemas respecto a la revisión por
  pares comentados en la sección \ref{intro}.
\end{enumerate}

\section{Science Distribution}

Toda interacción con la plataforma ha de realizarse mediante una cuenta Ethereum
(ver section \ref{tech:sec:ethereum}) y quedan grabadas en la blockchain de
este. Esto implicaría que a través de la dirección de un investigador científico
se puedan obtener dato de todos los papers que ha publicado y revisado. Todo la
comunidad científica podría sufrir una mejor gracias a este sistema. Además, los
nuevos investigadores que quieren empezar su carrera en el mundo académico
pueden obtener visibilidad si los revisores que revisan los papers que envían a
los Distributed Journals tienen alta reputación o no.

Si el sistema se implantara de manera exitosa, la comunidad científica empezaría
a cuestionarse la existencia de los publishers, y si estos desaparecieran, se
encontrarían nuevas formas de financiación de proyectos.

\section{Problems}
\figura{chart.png}{width=0.99\linewidth}{prob:txfee}{Ethereum transaction fees
  evolution}


Hay varios problemas para implantar este proyecto hoy en día, ya que las
tecnologías propuestas todavía tienen poda expansión y son poco conocidas por el
usuario medio.

La primera es el cambio de plataforma para los investigadores de la comunidad,
ya que la costumbre de utilizar las plataformas de hoy en día es dificil de
cambiar, por lo que proponer un cambio en los sistemas de comunicación y
revisión puede suponer un gran rechazo inicial, pudiendo llevar al proyecto a un
punto muerto, sobretodo si la metodología de conexión a la blockchain de
Ethereum es compleja actualmente.

La segunda es el precio de las transacciones. Ethereum es una moneda que fluctúa
bastante, y ultimamente se ha experimentado una gran crecida de todas los
precios de las criptomonedas con respecto a hace seis meses. Las subidas en el
precio provocan subidas en las transacciones, que a su vez provoca que la
interacción sea más cara para todos los usuarios que la utilizan.

\section{Papers Published}
Tenorio-Fornés, A., Jacynycz, V., Llop, D., Sánchez-Ruiz, A.A., Hassan, S. \textbf{“A
Decentralized Publication System for Open Science using Blockchain and IPFS”}.
\emph{PEERE International Conference on Peer Review Proceedings.} Rome (2018)


%%%
%%% Local Variables:
%%% mode: latex
%%% TeX-master: "../Tesis.tex"
%%% End:
