\chapter{State of the art}

\begin{FraseCelebre}
  \begin{Frase}
    The needs of the many outweigh the needs of the few
  \end{Frase}
  \begin{Fuente}
    Spock - The Wrath of Khan
  \end{Fuente}
\end{FraseCelebre}

% -------------------------------------------------------------------
% \section{Cooler Section}
% -------------------------------------------------------------------

Decentralized alternatives, despite their promises~\cite{bartlingblockchain},
are still in their infancy. A few proposals, none of them functional to date,
have appeared recently: a peer review proposal using
cryptocurrencies~\cite{tennant2017multi}, a blockchain-enabled app with voting
and storage of publications, again using
cryptocurrencies~\cite{morton2017aletheia}, or a peer review quality control
through blockchain-based cohort trainings~\cite{dhillon2016bench}. Additionally,
Ledger\footnote{https://ledgerjournal.org} journal records the publication
timestamps in the Bitcoin blockchain.

% Other platforms like Research Gate\footnote{https://www.researchgate.net},
% Mendeley \footnote{https://www.mendeley.com} or Academia
% \footnote{http://academia.edu} aim to create researcher social networks where
% they share OA papers, improving science dissemination.

This paper proposes the development of a decentralized publication system for
open science. It aims to challenge the technical infrastructure that supports
the middlemen role of traditional publishers. Due to the successes of the Opene
Access movement, some of the scientific knowledge is today freely provided by
the publishers. However, the content is still mostly served from their
infrastructure (i.e. servers, web platforms). This ownership of the
infrastructure gives them a power position over the scientific community which
produces the contents~\cite{fuster2010governance}. Such central and
oligopolistic position in science dissemination allows them to impose policies
(e.g. copyright ownership, Open Access prices) and concentrate profits.

The proposed system aims to move the infrastructure control from the publishers
to the scientific community. It entails the decentralization of three essential
functions of science dissemination: 1) the peer review process, 2) the selection
and recognition of peer reviewers, and 3) the distribution of scientific
knowledge. The following section provides an overview of the system features,
while the final section discusses its challenges.


%%%
%%% Local Variables:
%%% mode: latex
%%% TeX-master: "../Tesis.tex"
%%% End:
