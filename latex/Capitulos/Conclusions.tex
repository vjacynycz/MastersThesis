\chapter{Conclusions and future work}

\begin{FraseCelebre}
  \begin{Frase}
    We don't want to change. Every change is a menace to stability.
  \end{Frase}
  \begin{Fuente}
    Aldous Huxley - Brave New World
  \end{Fuente}
\end{FraseCelebre}

\section{Concluding remarks}

This work proposes the opening and decentralization of three of the peer review
and publication functions: 1) the peer review process communication, 2) the
reputation of reviewers, and 3) the distribution of papers and peer reviews. It
offers a first approach to the platform's implementation with a functional
prototype.

Distributed technologies such as Blockchain and IPFS may finally realize the
promise of Open Access, while enabling new models of science dissemination.

Opening and decentralizing the infrastructure enhances the transparency and
accountability of the system, and may provide a new arena to foster innovation.


The transparency provided by opening the peer review process allows the
construction of a reputation system of reviewers, but also raises concerns about
privacy and fairness. Besides, the introduction of a new public metric
(reviewers' reputation) may also affect researcher careers, adding pressure to
the already straining processes for academic survival~\cite{de2005publish}.

Blockchain technologies can be used to replicate the privacy settings currently
used in peer review processes. However, Blockchain can also be used to introduce
a new review model that supports the accountability of peer reviewing while
keeping the anonymity of blind and double blind reviews to improve fairness. The
implications of such accountable, open and anonymous review models are still to
be revealed.

Additionally, the system's infrastructure relies in new technologies with their
own challenges. Blockchain technologies face scalability, transaction costs,
inclusiveness and usability problems that remain open and under discussion. On
the other hand, distributed file systems such as IPFS may be more resilient, but
they still need somebody in charge of preserving and providing the data, since
without that responsible actor, it may result in unpredictable loss of content.

Other open issues that may be explored in future work are the exploration of
different copyright regimes, the challenging of traditional journal-centered
metrics to rate publication quality, different reputation algorithms, different
levels of openness, and the exploration of decentralized autonomous journals.

Despite the existing challenges, decentralizing the processes that Science
relies on could open up a whole new playing field, with implications we cannot
possibly foresee now. Will its benefits outweigh its risks?

\section{Future Work}

One of the first problems to attack would be the anonymity in the peer reviews,
because no matter how utopian it may seem, a completely public system in which
there is no anonymity of the reviewers and the authors is quite difficult to
implement. This problem could be mitigated with some of the solutions proposed
in the section~\ref{sec:privacyRating}. Nevertheless, Decentralized Science's
source code can be migrated to other blockchain technologies. Thus, if in the
future, there a similar technology to Ethereum, where anonymity protocols are
already implemented, a version of this platform could be easily implemented.

As extensions to this work, the following ideas are proposed:

\begin{itemize}
\item The incorporation of a system to forward papers already accepted, trying
  to expand the research they propose, including the possibility of being able
  to follow the research lines of other authors, completing their papers to
  build a collaborative scientific community, based on mutual support.
\item A standard citation methodology in which both papers and authors are
  Ethereum's addresses replacing identifiers like ISBN or DOI. An author address
  could give information about the author's work, offering the possibility to
  view the papers she sent, the ones she reviewed and collaborations made with
  other authors. A paper's address could give information about the impact
  factor it has, showing all other papers that cite it.
\item The automation of the assignment of reviewers for a newly submitted
  article, so that the formal role of the editor disappears, since the choice
  will depend entirely on the smart contract of the journal to which it
  corresponds. This choice could be based on the trust that the community has in
  the reviewers' reputation network, being able to eliminate one of the
  intermediaries from the system.
\item Finally, the intermediary of the journals could be eliminated. If the
  assignment of reviewers and publication of the articles is automatic and is in
  the blockchain, the existence of the journal could be questioned by the
  scientific community. With the support of a fully functional and widely used
  reputation system for reviewers, an ideal way of scientific publishing could
  be a large library in which authors submit their paper, reviewers are randomly
  chosen based on their reputation, and papers are automatically published.
\end{itemize}

To conclude this work, I propose a future PhD with new development lines with
higher impact in the scientific community. This proposal would include: An
exhaustive exploration of all blockchain technologies, interviews with real
users to validate the platform, social simulation using multi-agent systems to
analyze the platform's behavior and reaching potential users to deploy demos in
real environments.

Within a few years, scientific publication methods may be based on projects such
as the one presented in this work, changing the paradigm that has been imposed
for so many years in the scientific research process.

\cleardoublepage


\section{Conclusiones finales}

Este trabajo propone la descentralización de tres de las funciones del proceso
de revisión por pares: 1) la comunicación del proceso de revisión por pares, 2)
la reputación de los revisores, y 3) la distribución de los artículos y las
revisiones de estos. Ofrece un primer acercamiento a la implementación de la
plataforma con un prototipo funcional.

Las tecnologías distribuidas como Blockchain e IPFS pueden cumplir las promesas
del \ii{Open Access} a la vez que abren nuevos modelos para la distribución de
ciencia.

La descentralización de la estructura mejora la trasparencia y la honestidad del
sistema, y puede proporcionar nuevos escenarios para fomentar la innovación.
También se ha de tener en cuenta que el sistema y los prototipos propuestos no
se basan en el uso de criptomonedas, ya que se centran en un enfoque sin fines
lucrativos, lejos de los enfoques comerciales impulsados ​​por las startups, cada
vez más comunes dentro en el espacio blockchain.

La transparencia proporcionada al abrir el proceso de revisión por pares permite
la construcción de un sistema de reputación de revisores, pero también plantea
problemas sobre la privacidad y la equidad. Además, la introducción de una nueva
métrica pública (la reputación de los revisores) también puede afectar las
carreras de los investigadores, lo que puede suponer un aumento en la presión a
los procesos ya de por sí agotadores para la supervivencia
académica~\cite{de2005publish}.

Las tecnologías de Blockchain se pueden usar para replicar la configuración de
privacidad utilizada actualmente en los procesos de revisión por pares. Sin
embargo, Blockchain también se puede utilizar para presentar un nuevo modelo de
revisión que respalde la responsabilidad de la revisión por pares y al mismo
tiempo mantener el anonimato de las revisiones a ciegas y doble ciego para
mejorar la equidad. Las implicaciones de estos posibles modelos de revisión
responsables, abiertas y anónimas aún están por revelarse.


Además, la infraestructura del sistema se basa en nuevas tecnologías con sus
propios desafíos. Las tecnologías de Blockchain se enfrentan a problemas de
escalabilidad, costos de transacción, inclusión y usabilidad. Por otro lado, los
sistemas de archivos distribuidos como IPFS pueden ser más resistentes, pero aún
necesitan a alguien a cargo de preservar y proporcionar los datos, ya que sin
esa persona responsable, puede provocar la pérdida imprevista de contenido
almacenado por las plataformas que utilicen dicho sistema de archivos.

Otros temas abiertos que pueden explorarse en el trabajo futuro son la
exploración de diferentes modelos de derechos de autor, el desafío de las
métricas tradicionales centradas en revistas para calificar la calidad de la
publicación, diferentes algoritmos de reputación, diferentes niveles de apertura
y la exploración de revistas autónomas descentralizadas, capaces de operar de
manera automática sin interacción de los usuarios.

A pesar de los desafíos existentes, la descentralización de los procesos en los
que se basa la ciencia podría abrir un nuevo campo de exploración, con
implicaciones que posiblemente no podamos prever ahora. ¿Podrán los beneficios
superar a los riesgos?


\section{Trabajo futuro}

Uno de los primeros problemas a atacar sería el anonimato en las revisiones por
pares, porque no importa lo utópico que parezca, un sistema completamente
público en el que no hay anonimato de los revisores y los autores es difícil de
implementar. Este problema podría mitigarse con algunas de las soluciones
propuestas en la sección~\ref{sec:privacyRating}. Sin embargo, el código fuente
de este trabajo se puede migrar a otras tecnologías de blockchain. Por lo tanto,
si en el futuro, existe una tecnología similar a Ethereum, donde los protocolos
de anonimato ya están implementados, una versión de esta plataforma podría
implementarse fácilmente.


Como posibles extensiones de este proyecto se proponen las siguientes ideas:

\begin{itemize}
\item La incorporación de un sistema para enviar documentos ya aceptados,
  tratando de ampliar la investigación que proponen, incluyendo la posibilidad
  de seguir las líneas de investigación de otros autores, completando sus
  trabajos para construir una comunidad científica colaborativa, basada en el
  apoyo mutuo.
\item Una metodología de citación estándar en la que tanto documentos como
  autores son direcciones de Ethereum que reemplazan identificadores como ISBN o
  DOI. Una dirección de autor podría proporcionar información sobre el trabajo
  del autor, ofreciendo la posibilidad de ver los trabajos que envió, los que
  revisó y las colaboraciones realizadas con otros autores. La dirección de un
  artículos podría proporcionar información sobre el factor de impacto que
  tiene, mostrando todos los demás artículos que lo citan.
\item La automatización de la asignación de revisores para un artículo recién
  enviado, de modo que el papel formal del editor desaparezca, ya que la
  elección dependerá completamente del contrato inteligente de la revista a la
  que corresponda. Esta elección podría basarse en la confianza que la comunidad
  tiene en la red de reputación de los revisores, pudiendo eliminar a uno de los
  intermediarios del sistema.
\item Finalmente, el intermediario de las revistas científicas podría
  eliminarse. Si la asignación de revisores y la publicación de los artículos es
  automática y está en el blockchain, la comunidad científica cuestionaría la
  existencia de estas revistas. Con el apoyo de un sistema de reputación
  totalmente funcional y ampliamente utilizado para los revisores, una forma
  ideal de publicación científica podría ser una gran biblioteca en la que los
  autores presenten su trabajo, los revisores se elijan al azar según su
  reputación y los trabajos se publiquen automáticamente.
\end{itemize}

Para concluir este trabajo, propongo un futuro doctorado con nuevas líneas de
desarrollo con mayor impacto en la comunidad. Esta propuesta incluiría: una
exploración exhaustiva de todas las tecnologías de blockchain, entrevistas con
usuarios reales para validar la plataforma, simulación social usando sistemas
multi-agente para analizar el comportamiento de la plataforma y contactar con
usuarios potenciales para implementar demos en entornos reales.

Puede de dentro de unos años los métodos de publicación científica se basen en
proyectos como el que presenta este trabajo, cambiando el paradigma que lleva
tantos años impuesto en el proceso de investigación científica.


\section{Conferencias y artículos académicos}
\label{sec:conferences-papers}

El proyecto de Decentralized Science ha sido presentado en las siguientes conferencias:

\begin{itemize}
\item \bb{EthCC2018} - Ethereum community conference. \ii{París, Francia} (Marzo
  2018). \url{https://ethcc.io/}

\item \bb{PEERE2018} - International Conference on Peer Review. \ii{Roma, Italia}
  (Marzo 2018). \url{http://www.peere.org/conference/}

  \itbf{SPONBC2018} - Scientific Publishing on the Blockchain. \ii{Viena,
    Austria} (Mayo 2018).
  
  
  \url{https://www.blockchainforscience.com/2018/02/09/sponbc2018/}
\end{itemize}

Este proyecto también ha presentado los siguentes artículos académicos:

\begin{minipage}{\linewidth}
  Tenorio-Fornés, A., Jacynycz, V., Llop, D., Sánchez-Ruiz, A.A., Hassan, S.
  \textbf{``A Decentralized Publication System for Open Science using Blockchain
    and IPFS''}. \emph{PEERE International Conference on Peer Review
    Proceedings.} Rome (2018). Estado: publicado.
\end{minipage}

\vspace{5mm}

\begin{minipage}{\linewidth}
  Tenorio-Fornés, A., Jacynycz, V., Llop, D., Sánchez-Ruiz, A.A., Hassan, S.
  \textbf{``Towards a Decentralized Process for Scientific Publication and Peer
    Review using Blockchain and IPFS''}. \ii{HICSS Hawaii International
    Conference on System Sciences (CORE-A)}. Hawaii (2019). Estado: aceptado.
\end{minipage}


\section{Conferences and Papers}
\label{sec:conferences-papers}

Decentralized Science project has been presented at the following conferences:

\begin{itemize}
\item \bb{EthCC2018} - Ethereum community conference. \ii{Paris, France} (March
  2018). \url{https://ethcc.io/}

\item \bb{PEERE2018} - International Conference on Peer Review. \ii{Rome, Italy}
  (March 2018). \url{http://www.peere.org/conference/}

  \itbf{SPONBC2018} - Scientific Publishing on the Blockchain. \ii{Vienna,
    Austria} (May 2018).
  
  
  \url{https://www.blockchainforscience.com/2018/02/09/sponbc2018/}
\end{itemize}

This project also has presented the following papers:

\begin{minipage}{\linewidth}
  Tenorio-Fornés, A., Jacynycz, V., Llop, D., Sánchez-Ruiz, A.A., Hassan, S.
  \textbf{``A Decentralized Publication System for Open Science using Blockchain
    and IPFS''}. \emph{PEERE International Conference on Peer Review
    Proceedings.} Rome (2018). Status: Published.
\end{minipage}

\vspace{5mm}

\begin{minipage}{\linewidth}
  Tenorio-Fornés, A., Jacynycz, V., Llop, D., Sánchez-Ruiz, A.A., Hassan, S.
  \textbf{``Towards a Decentralized Process for Scientific Publication and Peer
    Review using Blockchain and IPFS''}. \ii{HICSS Hawaii International
    Conference on System Sciences (CORE-A)}. Hawaii (2019). Status: Accepted.
\end{minipage}

%%%
%%% Local Variables:
%%% mode: latex
%%% TeX-master: "../Tesis.tex"
%%% End:

