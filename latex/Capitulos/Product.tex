\chapter{Proof of concept}
\label{poc}
\begin{FraseCelebre}
  \begin{Frase}
    Sooner or later you're going to realize just as I did that there's a
    difference between knowing the path and walking the path.
  \end{Frase}
  \begin{Fuente}
    Morpheus - The Matrix
  \end{Fuente}
\end{FraseCelebre}

\section{Platform showcase}

As proof of concept, this work provides several files to run and test a local
server with the prototype of Decentralized Science. The server consists of a web
interface that allows users to access to the functionality of the distributed
system (IFPS + Ethereum) without the need to know these technologies.

Users can interact with this platform and connect to a blockchain test net in
which all the accounts have unlimited money to make all the transactions to
interact with Ethereum.

This prototype is a decentralized system, users may run a local node instead of
connecting to the gateway provided. The source code of the platform is open
source, and can be obtained and deployed in an Ethereum and IPFS node. Figure
\ref{poc:mockup} shows a mockup to illustrate the platform's structure and
rotation among its contents.

It should be noted that this web framework acts as a layer between blockchain
and user interaction and it can be modified by any user since the internal
design of the platform is inside blockchain.

\figuraHere{mockup.png} {width=0.9\linewidth}{Mock-up of the possible
  interactions of the platform's front-end}{poc:mockup}


\sst{Front-page}

As showed in the figure~\ref{poc:fp}, the front page of the platform shows three
main sections:

\begin{itemize}
  \itbf{A showcase showing the journals registered:} All journals registered in
  the platform are automatically showed in the front page in order of
  popularity. Showing the most popular journals first (i.e. journals with the
  most publications).

  \itbf{Register journal link:} A link to the journal registration page.

\end{itemize}

The platform also has a searchbox to search queries within the platform.

\figuraHere{homepage.png} {width=0.9\linewidth}{The front page of the
  platform}{poc:fp}

\sst{Register Journal}

Decentralized Science offers the possibility to register journals within the
platform. Journal editors who want to adopt this system, can implement a smart
contract template hosted in github, upload it to Ethereum and provide the
address the registration page. This will create a transaction in Ethereum and
will link the new registered journal in the platform.

\figuraHere{register.png}%
{width=0.9\linewidth}%
{Register journal page}%
{poc:rj}

The figure~\ref{poc:rj} shows the registration page for journals. The HTML form
creates an Ethereum transaction to Decentralized Science's address. The address
provided must fit decentralized journal template, otherwise the transaction will
be unsuccessful. In this prototype the journal smart contract programming and
deployment should be done outside this platform following the template provided,
but in future implementations this feature will be inside the framework.

\sst{Journal Page}

The journal page contains all the information about the journal:
\begin{itemize}
\item Title of the journal.
\item A short description about the journal.
\item An image of the journals cover.
\item Latest papers published and submitted (preprints) with a corresponding
  link to IPDF.
 \item A list to all the papers of the  journal.
\item A link to the paper submission page (see figure~\ref{poc:subm}).
\item The average review time of the papers published in the journal.
\end{itemize}

\figuraHere{journal.png}%
{width=0.9\linewidth}%
{Journal page}%
{poc:j}

\figuraHere{submission.png} {width=0.9\linewidth}{Submission page}{poc:subm}

The figure~\ref{poc:j} shows the journal's page with all the information
mentioned above. The HTML connects to the journal's Ethereum address to get the
IPFS address of the journal. This last address references a \emph{JSON} file
containing all the information about the journal.

% \vfill{1em}

\sst{Paper page}

A paper page contains useful information about the paper:
\begin{itemize}
\item Title.
\item Authors, with links to each author page.
\item Abstact of the paper.
\item A download link for the paper and links to download each draft, offering
  the possibillity to view the previous work of the researchers.
\item The reviews from the reviewers showing the author, the IPFS link with the text of the review
  and the acceptance level. Each review can also be rated if the user's address
  can send  the rating.
\end{itemize}
\figuraHere{paper.png}%
{width=0.9\linewidth}%
{Paper page}%
{poc:p}

The figure~\ref{poc:p} shows the paper's page with all the information mentioned
above.

\sst{Author's page}

Each time a new author submits a paper to a journal its address is registered.
The authors page shows all the information about a researcher, connecting to the
journal contract and the reputation contract. The main features about this page
are:

\begin{itemize}
  \itbf{Reputation:} The reputation of a researcher is represented with a 5-star
  rating, depending on how good or bad is as a reviewer. This star rating is
  calculated by the HTML depending on the reputation score (see section~\ref{sec:distr-revi-reput}),
  transforming 0 to 1 score into 0 to 5 stars. It also contains the percent of
  the reviews delivered in time.

  \itbf{Ethereum's address:} The address of the researcher can be used to verify
  all the transactions done with that address. This offers the possibility to
  check a researcher career with only its address.

  \itbf{Papers published:} All the papers published as a researcher with the
  mentioned address.

  \itbf{Papers reviewed:} All the reviews the researcher has done. 
\end{itemize}

\figuraHere{rating.png}%
{width=0.97\linewidth}%
{Author's page}%
{poc:a}

The figure~\ref{poc:a} shows the author's page with all the information
mentioned above.

\sst{Journal's management page}

The platforms allows addresses with editor privileges to assign reviewers to
pending papers. The page shows the pending, accepted an rejected papers as shown
in the figure~\ref{poc:jmp}.

For each pending paper there is a modal window to assign the addresses of the reviewers
as shown in figure~\ref{poc:am}. This will create an Ethereum transaction to
make the assignment public, and allows the reviewers assigned to accept the
review.

\figuraHere{manage.png}%
{width=0.97\linewidth}%
{Journal's management page}%
{poc:jmp}

\figuraHere{assign.png}%
{width=0.97\linewidth}%
{Assign modal window from journal's management page}%
{poc:am}


%%% Local Variables:
%%% mode: latex
%%% TeX-master: "../Tesis.tex"
%%% End:
