%---------------------------------------------------------------------
%
%                          TeXiS_toc.tex
%
%---------------------------------------------------------------------
%
% TeXiS_toc.tex
% Copyright 2009 Marco Antonio Gomez-Martin, Pedro Pablo Gomez-Martin
%
% This file belongs to TeXiS, a LaTeX template for writting
% Thesis and other documents. The complete last TeXiS package can
% be obtained from http://gaia.fdi.ucm.es/projects/texis/
%
% This work may be distributed and/or modified under the
% conditions of the LaTeX Project Public License, either version 1.3
% of this license or (at your option) any later version.
% The latest version of this license is in
%   http://www.latex-project.org/lppl.txt
% and version 1.3 or later is part of all distributions of LaTeX
% version 2005/12/01 or later.
%
% This work has the LPPL maintenance status `maintained'.
%
% The Current Maintainers of this work are Marco Antonio Gomez-Martin
% and Pedro Pablo Gomez-Martin
%
%---------------------------------------------------------------------
%
% Contiene  los  comandos  para  generar los  _ndices  del  documento,
% entendiendo por _ndices las tablas de contenidos.
%
% Genera  el  _ndice normal  ("tabla  de  contenidos"),  el _ndice  de
% figuras y el de tablas. Tambi_n  crea "marcadores" en el caso de que
% se est_ compilando con pdflatex para que aparezcan en el PDF.
%
%---------------------------------------------------------------------


% Primero un poquito de configuraci_n...


% Pedimos que inserte todos los ep_grafes hasta el nivel \subsection en
% la tabla de contenidos.
\setcounter{tocdepth}{2}

% Le  pedimos  que nos  numere  todos  los  ep_grafes hasta  el  nivel
% \subsubsection en el cuerpo del documento.
\setcounter{secnumdepth}{3}


% Creamos los diferentes _ndices.

% Lo primero un  poco de trabajo en los marcadores  del PDF. No quiero
% que  salga una  entrada  por cada  _ndice  a nivel  0...  si no  que
% aparezca un marcador "_ndices", que  tenga dentro los otros tipos de
% _ndices.  Total, que creamos el marcador "_ndices".
% Antes de  la creaci_n  de los _ndices,  se a_aden los  marcadores de
% nivel 1.

\ifpdf
   \pdfbookmark{Indexes}{indexes}
\fi

% Tabla de contenidos.
%
% La  inclusi_n  de '\tableofcontents'  significa  que  en la  primera
% pasada  de  LaTeX  se  crea   un  fichero  con  extensi_n  .toc  con
% informaci_n sobre la tabla de contenidos (es conceptualmente similar
% al  .bbl de  BibTeX, creo).  En la  segunda ejecuci_n  de  LaTeX ese
% documento se utiliza para  generar la verdadera p_gina de contenidos
% usando la  informaci_n sobre los  cap_tulos y dem_s guardadas  en el
% .toc
\ifpdf
   \pdfbookmark[1]{Table of contents}{table of contents}
\fi

\cabeceraEspecial{Index}

\tableofcontents

\newpage

% _ndice de figuras
%
% La idea es semejante que para  el .toc del _ndice, pero ahora se usa
% extensi_n .lof (List Of Figures) con la informaci_n de las figuras.



\ifpdf
   \pdfbookmark[1]{Index of figures}{index of figures}
\fi

\cabeceraEspecial{Index of Figures}

\listoffigures

\newpage

% _ndice de tablas
% Como antes, pero ahora .lot (List Of Tables)

\ifpdf
   \pdfbookmark[1]{Index of tables}{index of tables}
\fi

\cabeceraEspecial{Index of tables}

\listoftables

\newpage

% Variable local para emacs, para  que encuentre el fichero maestro de
% compilaci_n y funcionen mejor algunas teclas r_pidas de AucTeX

%%%
%%% Local Variables:
%%% mode: latex
%%% TeX-master: "../Tesis.tex"
%%% End:
