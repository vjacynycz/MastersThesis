%---------------------------------------------------------------------
%
%                        TeXiS_acron.tex
%
%---------------------------------------------------------------------
%
% TeXiS_acron.tex
% Copyright 2009 Marco Antonio Gomez-Martin, Pedro Pablo Gomez-Martin
%
% This file belongs to TeXiS, a LaTeX template for writting
% Thesis and other documents. The complete last TeXiS package can
% be obtained from http://gaia.fdi.ucm.es/projects/texis/
%
% This work may be distributed and/or modified under the
% conditions of the LaTeX Project Public License, either version 1.3
% of this license or (at your option) any later version.
% The latest version of this license is in
%   http://www.latex-project.org/lppl.txt
% and version 1.3 or later is part of all distributions of LaTeX
% version 2005/12/01 or later.
%
% This work has the LPPL maintenance status `maintained'.
%
% The Current Maintainers of this work are Marco Antonio Gomez-Martin
% and Pedro Pablo Gomez-Martin
%
%---------------------------------------------------------------------
%
% Contiene  los  comandos  para  generar  el listado de acr_nimos
% documento.
%
%---------------------------------------------------------------------
%
% NOTA IMPORTANTE:  para que la  generaci_n de acr_nimos  funcione, al
% menos  debe  existir  un  acr_nimo   en  el  documento.  Si  no,  la
% compilaci_n  del   fichero  LaTeX  falla  con   un  error  "extra_o"
% (indicando  que  quiz_  falte  un \item).   Consulta  el  comentario
% referente al paquete glosstex en TeXiS_pream.tex.
%
%---------------------------------------------------------------------


% Redefinimos a espa_ol  el t_tulo de la lista  de acr_nimos (Babel no
% lo hace por nosotros esta vez)

\def\listacronymname{Lista de acr_nimos}

% Para el glosario:
% \def\glosarryname{Glosario}

% Si se  va a generar  la tabla de  contenidos (el _ndice  habitual) y
% tambi_n vamos a  generar la lista de acr_nimos  (ambas decisiones se
% toman en  funci_n de  la definici_n  o no de  un par  de constantes,
% puedes consultar config.tex  para m_s informaci_n), entonces metemos
% en la  tabla de contenidos una  entrada para marcar  la p_gina donde
% est_ el _ndice de palabras.

\ifx\generatoc\undefined
\else
   \addcontentsline{toc}{chapter}{\listacronymname}
\fi


% Generamos la lista de acr_nimos (en realidad el _ndice asociado a la
% lista "acr" de GlossTeX)

\printglosstex(acr)

% Variable local para emacs, para  que encuentre el fichero maestro de
% compilaci_n y funcionen mejor algunas teclas r_pidas de AucTeX

%%%
%%% Local Variables:
%%% mode: latex
%%% TeX-master: "../Tesis.tex"
%%% End:
