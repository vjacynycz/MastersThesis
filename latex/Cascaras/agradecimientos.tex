%---------------------------------------------------------------------
%
%                      agradecimientos.tex
%
%---------------------------------------------------------------------
%
% agradecimientos.tex
% Copyright 2009 Marco Antonio Gomez-Martin, Pedro Pablo Gomez-Martin
%
% This file belongs to the TeXiS manual, a LaTeX template for writting
% Thesis and other documents. The complete last TeXiS package can
% be obtained from http://gaia.fdi.ucm.es/projects/texis/
%
% Although the TeXiS template itself is distributed under the
% conditions of the LaTeX Project Public License
% (http://www.latex-project.org/lppl.txt), the manual content
% uses the CC-BY-SA license that stays that you are free:
%
%    - to share & to copy, distribute and transmit the work
%    - to remix and to adapt the work
%
% under the following conditions:
%
%    - Attribution: you must attribute the work in the manner
%      specified by the author or licensor (but not in any way that
%      suggests that they endorse you or your use of the work).
%    - Share Alike: if you alter, transform, or build upon this
%      work, you may distribute the resulting work only under the
%      same, similar or a compatible license.
%
% The complete license is available in
% http://creativecommons.org/licenses/by-sa/3.0/legalcode
%
%---------------------------------------------------------------------
%
% Contiene la pígina de agradecimientos.
%
% Se crea como un capítulo sin numeraciín.
%
%---------------------------------------------------------------------

\chapter{Agradecimientos}
\cabeceraEspecial{Agradecimientos}

\begin{FraseCelebre}
\begin{Frase}
I find your lack of faith disturbing\end{Frase}
\begin{Fuente}
Darth Vader, Star Wars: A New Hope.
\end{Fuente}
\end{FraseCelebre}

Me gustaría agradecer a los profesores Samer Hassan collado y Antonio Sanchez Ruiz-Granados
por el apoyo prestato para realizar este trabajo. 
\\
\mbox{}
\\
Al departamento de ISIA por fomentar mis ideas y pontenciarlas.
\\
\mbox{}
\\
A la universidad complutense por brindarme el conocimiento necesario para poder
realizar las ideas necesarias.
\\
\mbox{}
\\
A todos los que piensan que este trabajo no llegará a ningun lado, que me dan
fuerzas para seguir adelante.
\endinput
% Variable local para emacs, para  que encuentre el fichero maestro de
% compilaciín y funcionen mejor algunas teclas rípidas de AucTeX
%%%
%%% Local Variables:
%%% mode: latex
%%% TeX-master: "../Tesis.tex"
%%% End:
