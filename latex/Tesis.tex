
\documentclass[11pt,a4paper,twoside]{book}

%
% Definimos  el   comando  \compilaCapitulo,  que   luego  se  utiliza
% (opcionalmente) en config.tex. Quedar_a  mejor si tambi_n se definiera
% en  ese fichero,  pero por  el modo  en el  que funciona  eso  no es
% posible. Puedes consultar la documentaci_n de ese fichero para tener
% m_s  informaci_n. Definimos tambi_n  \compilaApendice, que  tiene el
% mismo  cometido, pero  que se  utiliza para  compilar  _nicamente un
% ap_ndice.
%
%
% Si  queremos   compilar  solo   una  parte  del   documento  podemos
% especificar mediante  \includeonly{...} qu_ ficheros  son los _nicos
% que queremos  que se incluyan.  Esto  es _til por  ejemplo para s_lo
% compilar un cap_tulo.
%
% El problema es que todos aquellos  ficheros que NO est_n en la lista
% NO   se  incluir_n...  y   eso  tambi_n   afecta  a   ficheros  de
% la plantilla...
%
% Total,  que definimos  una constante  con los  ficheros  que siempre
% vamos a querer compilar  (aquellos relacionados con configuraci_n) y
% luego definimos \compilaCapitulo.
\newcommand{\ficherosBasicosTeXiS}{%
TeXiS/TeXiS_pream,TeXiS/TeXiS_cab,TeXiS/TeXiS_bib,TeXiS/TeXiS_cover,%
TeXiS/TeXiS_part%
}
\newcommand{\ficherosBasicosTexto}{%
constantes,guionado,Cascaras/bibliografia,config%
}
\newcommand{\compilaCapitulo}[1]{%
\includeonly{\ficherosBasicosTeXiS,\ficherosBasicosTexto,Capitulos/#1}
}

\newcommand{\compilaApendice}[1]{%
\includeonly{\ficherosBasicosTeXiS,\ficherosBasicosTexto,Apendices/#1}
}

%---------------------------------------------------------------------
%
%                          config.tex
%
%---------------------------------------------------------------------
%
% Contiene la  definici_n de constantes  que determinan el modo  en el
% que se compilar_ el documento.
%
%---------------------------------------------------------------------
%
% En concreto, podemos  indicar si queremos "modo release",  en el que
% no  aparecer_n  los  comentarios  (creados  mediante  \com{Texto}  o
% \comp{Texto}) ni los "por  hacer" (creados mediante \todo{Texto}), y
% s_ aparecer_n los _ndices. El modo "debug" (o mejor dicho en modo no
% "release" muestra los _ndices  (construirlos lleva tiempo y son poco
% _tiles  salvo  para   la  versi_n  final),  pero  s_   el  resto  de
% anotaciones.
%
% Si se compila con LaTeX (no  con pdflatex) en modo Debug, tambi_n se
% muestran en una esquina de cada p_gina las entradas (en el _ndice de
% palabras) que referencian  a dicha p_gina (consulta TeXiS_pream.tex,
% en la parte referente a show).
%
% El soporte para  el _ndice de palabras en  TeXiS es embrionario, por
% lo  que no  asumas que  esto funcionar_  correctamente.  Consulta la
% documentaci_n al respecto en TeXiS_pream.tex.
%
%
% Tambi_n  aqu_ configuramos  si queremos  o  no que  se incluyan  los
% acr_nimos  en el  documento final  en la  versi_n release.  Para eso
% define (o no) la constante \acronimosEnRelease.
%
% Utilizando \compilaCapitulo{nombre}  podemos tambi_n especificar qu_
% cap_tulo(s) queremos que se compilen. Si no se pone nada, se compila
% el documento  completo.  Si se pone, por  ejemplo, 01Introduccion se
% compilar_ _nicamente el fichero Capitulos/01Introduccion.tex
%
% Para compilar varios  cap_tulos, se separan sus nombres  con comas y
% no se ponen espacios de separaci_n.
%
% En realidad  la macro \compilaCapitulo  est_ definida en  el fichero
% principal tesis.tex.
%
%---------------------------------------------------------------------


% Comentar la l_nea si no se compila en modo release.
% TeXiS har_ el resto.
% ___Si cambias esto, haz un make clean antes de recompilar!!!
\def\release{1}


% Descomentar la linea si se quieren incluir los
% acr_nimos en modo release (en modo debug
% no se incluir_n nunca).
% ___Si cambias esto, haz un make clean antes de recompilar!!!
%\def\acronimosEnRelease{1}


% Descomentar la l_nea para establecer el cap_tulo que queremos
% compilar

% \compilaCapitulo{01Introduccion}
% \compilaCapitulo{02EstructuraYGeneracion}
% \compilaCapitulo{03Edicion}
% \compilaCapitulo{04Imagenes}
% \compilaCapitulo{05Bibliografia}
% \compilaCapitulo{06Makefile}

% \compilaApendice{01AsiSeHizo}

% Variable local para emacs, para  que encuentre el fichero maestro de
% compilaci_n y funcionen mejor algunas teclas r_pidas de AucTeX
%%%
%%% Local Variables:
%%% mode: latex
%%% TeX-master: "./Tesis.tex"
%%% End:


% Paquete de la plantilla

\usepackage{TeXiS/TeXiS}
\usepackage[lmargin=3.5cm,rmargin=2.5cm]{geometry}
\setcitestyle{square}


% Incluimos el fichero con comandos de constantes
%---------------------------------------------------------------------
%
%                          constantes.tex
%
%---------------------------------------------------------------------
%
% Fichero que  declara nuevos comandos LaTeX  sencillos realizados por
% comodidad en la escritura de determinadas palabras
%
%---------------------------------------------------------------------

%%%%%%%%%%%%%%%%%%%%%%%%%%%%%%%%%%%%%%%%%%%%%%%%%%%%%%%%%%%%%%%%%%%%%%
% Comando: 
%
%       \titulo
%
% Resultado: 
%
% Escribe el t�tulo del documento.
%%%%%%%%%%%%%%%%%%%%%%%%%%%%%%%%%%%%%%%%%%%%%%%%%%%%%%%%%%%%%%%%%%%%%%
\def\titulo{Blockchain-based reputation system for peer reviewing}

%%%%%%%%%%%%%%%%%%%%%%%%%%%%%%%%%%%%%%%%%%%%%%%%%%%%%%%%%%%%%%%%%%%%%%
% Comando: 
%
%       \autor
%
% Resultado: 
%
% Escribe el autor del documento.
%%%%%%%%%%%%%%%%%%%%%%%%%%%%%%%%%%%%%%%%%%%%%%%%%%%%%%%%%%%%%%%%%%%%%%
\def\autor{Viktor Jacynycz García}

% Variable local para emacs, para  que encuentre el fichero maestro de
% compilaci�n y funcionen mejor algunas teclas r�pidas de AucTeX

%%%
%%% Local Variables:
%%% mode: latex
%%% TeX-master: "tesis.tex"
%%% End:
%también


\typeout{Copyright Marco Antonio and Pedro Pablo Gomez Martin}

%
% "Metadatos" para el PDF
%
\ifpdf\hypersetup{%
    pdftitle = {\titulo},
    pdfsubject = {Master's Thesis},
    pdfkeywords = {blockchain},
    pdfauthor = {\textcopyright\ \autor},
    pdfcreator = {\LaTeX\ con el paquete \flqq hyperref\frqq},
    pdfproducer = {pdfeTeX-0.\the\pdftexversion\pdftexrevision},
    }
    \pdfinfo{/CreationDate (\today)}
\fi


%- - - - - - - - - - - - - - - - - - - - - - - - - - - - - - - - - - -
%                        Documento
%- - - - - - - - - - - - - - - - - - - - - - - - - - - - - - - - - - -
\begin{document}

% Incluimos el  fichero de definici_n de guionado  de algunas palabras
% que LaTeX no ha dividido como deber_a
\input{guionado}

% Marcamos  el inicio  del  documento para  la  numeraci_n de  p_ginas
% (usando n_meros romanos para esta primera fase).
\frontmatter


%---------------------------------------------------------------------
%
%                          configCover.tex
%
%---------------------------------------------------------------------
%
% cover.tex
% Copyright 2009 Marco Antonio Gomez-Martin, Pedro Pablo Gomez-Martin
%
% This file belongs to the TeXiS manual, a LaTeX template for writting
% Thesis and other documents. The complete last TeXiS package can
% be obtained from http://gaia.fdi.ucm.es/projects/texis/
%
% Although the TeXiS template itself is distributed under the
% conditions of the LaTeX Project Public License
% (http://www.latex-project.org/lppl.txt), the manual content
% uses the CC-BY-SA license that stays that you are free:
%
%    - to share & to copy, distribute and transmit the work
%    - to remix and to adapt the work
%
% under the following conditions:
%
%    - Attribution: you must attribute the work in the manner
%      specified by the author or licensor (but not in any way that
%      suggests that they endorse you or your use of the work).
%    - Share Alike: if you alter, transform, or build upon this
%      work, you may distribute the resulting work only under the
%      same, similar or a compatible license.
%
% The complete license is available in
% http://creativecommons.org/licenses/by-sa/3.0/legalcode
%
%---------------------------------------------------------------------
%
% Fichero que contiene la configuración de la portada y de la
% primera hoja del documento.
%
%---------------------------------------------------------------------


% Pueden configurarse todos los elementos del contenido de la portada
% utilizando comandos.

%%%%%%%%%%%%%%%%%%%%%%%%%%%%%%%%%%%%%%%%%%%%%%%%%%%%%%%%%%%%%%%%%%%%%%
% Tótulo del documento:
% \tituloPortada{titulo}
% Nota:
% Si no se define se utiliza el del \titulo. Este comando permite
% cambiar el tótulo de forma que se especifiquen dónde se quieren
% los retornos de carro cuando se utilizan fuentes grandes.
%%%%%%%%%%%%%%%%%%%%%%%%%%%%%%%%%%%%%%%%%%%%%%%%%%%%%%%%%%%%%%%%%%%%%%

\tituloPortada{Blockchain-based reputation system for peer reviewing}

%%%%%%%%%%%%%%%%%%%%%%%%%%%%%%%%%%%%%%%%%%%%%%%%%%%%%%%%%%%%%%%%%%%%%%
% Autor del documento:
% \autorPortada{Nombre}
% Se utiliza en la portada y en el valor por defecto del
% primer subtótulo de la segunda portada.
%%%%%%%%%%%%%%%%%%%%%%%%%%%%%%%%%%%%%%%%%%%%%%%%%%%%%%%%%%%%%%%%%%%%%%
\autorPortada{Viktor Jacynycz García}

%%%%%%%%%%%%%%%%%%%%%%%%%%%%%%%%%%%%%%%%%%%%%%%%%%%%%%%%%%%%%%%%%%%%%%
% Fecha de publicación:
% \fechaPublicacion{Fecha}
% Puede ser vacóo. Aparece en la óltima lónea de ambas portadas
%%%%%%%%%%%%%%%%%%%%%%%%%%%%%%%%%%%%%%%%%%%%%%%%%%%%%%%%%%%%%%%%%%%%%%
\fechaPublicacion{2017/2018 academic year}

%%%%%%%%%%%%%%%%%%%%%%%%%%%%%%%%%%%%%%%%%%%%%%%%%%%%%%%%%%%%%%%%%%%%%%
% Imagen de la portada (y escala)
% \imagenPortada{Fichero}
% \escalaImagenPortada{Numero}
% Si no se especifica, se utiliza la imagen TODO.pdf
%%%%%%%%%%%%%%%%%%%%%%%%%%%%%%%%%%%%%%%%%%%%%%%%%%%%%%%%%%%%%%%%%%%%%
\imagenPortada{Imagenes/Vectorial/escudoUCM}
\escalaImagenPortada{0.2}

%%%%%%%%%%%%%%%%%%%%%%%%%%%%%%%%%%%%%%%%%%%%%%%%%%%%%%%%%%%%%%%%%%%%%%
% Tipo de documento.
% \tipoDocumento{Tipo}
% Para el texto justo debajo del escudo.
% Si no se indica, se utiliza "TESIS DOCTORAL".
%%%%%%%%%%%%%%%%%%%%%%%%%%%%%%%%%%%%%%%%%%%%%%%%%%%%%%%%%%%%%%%%%%%%%%
\tipoDocumento{Master's Thesis}

%%%%%%%%%%%%%%%%%%%%%%%%%%%%%%%%%%%%%%%%%%%%%%%%%%%%%%%%%%%%%%%%%%%%%%
% Institución/departamento asociado al documento.
% \institucion{Nombre}
% Puede tener varias lóneas. Se utiliza en las dos portadas.
% Si no se indica apareceró vacóo.
%%%%%%%%%%%%%%%%%%%%%%%%%%%%%%%%%%%%%%%%%%%%%%%%%%%%%%%%%%%%%%%%%%%%%%
\institucion{
Máster en ingeniería informática \leavevmode \\[0.3em]
Master's degree in software engineering \leavevmode \\[0.3em]
Facultad de Informática \leavevmode \\[0.3em]
Universidad Complutense de Madrid
}

%%%%%%%%%%%%%%%%%%%%%%%%%%%%%%%%%%%%%%%%%%%%%%%%%%%%%%%%%%%%%%%%%%%%%%
% Director del trabajo.
% \directorPortada{Nombre}
% Se utiliza para el valor por defecto del segundo subtótulo, donde
% se indica quión es el director del trabajo.
% Si se fuerza un subtótulo distinto, no hace falta definirlo.
%%%%%%%%%%%%%%%%%%%%%%%%%%%%%%%%%%%%%%%%%%%%%%%%%%%%%%%%%%%%%%%%%%%%%%
\directorPortada{Samer Hassan Collado and Antonio Sánchez Ruiz-Granados}

%                      %%%%%%%%%%%%%%%%%%%%%%%%%%%%%%%%%%%%%%%%%%%%%%%%%%%%%%%%%%%%%%%%%%%%%
% Texto del primer subtótulo de la segunda portada.
% \textoPrimerSubtituloPortada{Texto}
% Para configurar el primer "texto libre" de la segunda portada.
% Si no se especifica se indica "Memoria que presenta para optar al
% tótulo de Doctor en Informótica" seguido del \autorPortada.
%%%%%%%%%%%%%%%%%%%%%%%%%%%%%%%%%%%%%%%%%%%%%%%%%%%%%%%%%%%%%%%%%%%%%%
%\textoPrimerSubtituloPortada{%
%\textit{Informe técnico del departamento}  \\ [0.3em]
%\textbf{Ingeniería del Software e Inteligencia Artificial} \\ [0.3em]
%\textbf{IT/2009/3}
%}

%%%%%%%%%%%%%%%%%%%%%%%%%%%%%%%%%%%%%%%%%%%%%%%%%%%%%%%%%%%%%%%%%%%%%%
% Texto del segundo subtótulo de la segunda portada.
%
%\textoSegundoSubtituloPortada{curso 2017/2018}% Para configurar el segundo "texto libre" de la segunda portada.
% Si no se especifica se indica "Dirigida por el Doctor" seguido
% del

%\directorPortada.
%%%%%%%%%%%%%%%%%%%%%%%%%%%%%%%%%%%%%%%%%%%%%%%%%%%%%%%%%%%%%%%%%%%%%%
%\textoSegundoSubtituloPortada{%
%\textit{Versión \texisVer}
%}

%%%%%%%%%%%%%%%%%%%%%%%%%%%%%%%%%%%%%%%%%%%%%%%%%%%%%%%%%%%%%%%%%%%%%%
% \explicacionDobleCara
% Si se utiliza, se aclara que el documento estó preparado para la
% impresión a doble cara.
%%%%%%%%%%%%%%%%%%%%%%%%%%%%%%%%%%%%%%%%%%%%%%%%%%%%%%%%%%%%%%%%%%%%%%
\explicacionDobleCara

%%%%%%%%%%%%%%%%%%%%%%%%%%%%%%%%%%%%%%%%%%%%%%%%%%%%%%%%%%%%%%%%%%%%%%
% \isbn
% Si se utiliza, apareceró el ISBN detrós de la segunda portada.
%%%%%%%%%%%%%%%%%%%%%%%%%%%%%%%%%%%%%%%%%%%%%%%%%%%%%%%%%%%%%%%%%%%%%%
%\isbn{978-84-692-7109-4}


%%%%%%%%%%%%%%%%%%%%%%%%%%%%%%%%%%%%%%%%%%%%%%%%%%%%%%%%%%%%%%%%%%%%%%
% \copyrightInfo
% Si se utiliza, apareceró información de los derechos de copyright
% detrós de la segunda portada.
%%%%%%%%%%%%%%%%%%%%%%%%%%%%%%%%%%%%%%%%%%%%%%%%%%%%%%%%%%%%%%%%%%%%%%
\copyrightInfo{\autor}

\makeCover




%---------------------------------------------------------------------
%
%                      dedicatoria.tex
%
%---------------------------------------------------------------------
%
% dedicatoria.tex
% Copyright 2009 Marco Antonio Gomez-Martin, Pedro Pablo Gomez-Martin
%
% This file belongs to the TeXiS manual, a LaTeX template for writting
% Thesis and other documents. The complete last TeXiS package can
% be obtained from http://gaia.fdi.ucm.es/projects/texis/
%
% Although the TeXiS template itself is distributed under the
% conditions of the LaTeX Project Public License
% (http://www.latex-project.org/lppl.txt), the manual content
% uses the CC-BY-SA license that stays that you are free:
%
%    - to share & to copy, distribute and transmit the work
%    - to remix and to adapt the work
%
% under the following conditions:
%
%    - Attribution: you must attribute the work in the manner
%      specified by the author or licensor (but not in any way that
%      suggests that they endorse you or your use of the work).
%    - Share Alike: if you alter, transform, or build upon this
%      work, you may distribute the resulting work only under the
%      same, similar or a compatible license.
%
% The complete license is available in
% http://creativecommons.org/licenses/by-sa/3.0/legalcode
%
%---------------------------------------------------------------------
%
% Contiene la página de dedicatorias.
%
%---------------------------------------------------------------------

\dedicatoriaUno{%

}

\dedicatoriaDos{%
\emph{%
  A mi madre, por ser una mujer con super poderes capaz de afrontar cualquier
  dificultad, siempre has estado ahi \\%
  \mbox{ }\\%
  A mi padre, por ayudarme a ser como soy ahora, nunca me rendiré contigo\\%
  \mbox{ }\\%
  A mi hermano, por ser el mejor, haces que quiera seguir superándome para alcanzarte\\%
  \mbox{ }\\%
  A Jenny, por cambiarme la vida, siempre tiras de mi cuando lo necesito\\% 
}%
}

\makeDedicatorias

% Variable local para emacs, para que encuentre el fichero
% maestro de compilación
%%%
%%% Local Variables:
%%% mode: latex
%%% TeX-master: "../Tesis.tex"
%%% End:


%---------------------------------------------------------------------
%
%                      agradecimientos.tex
%
%---------------------------------------------------------------------
%
% agradecimientos.tex
% Copyright 2009 Marco Antonio Gomez-Martin, Pedro Pablo Gomez-Martin
%
% This file belongs to the TeXiS manual, a LaTeX template for writting
% Thesis and other documents. The complete last TeXiS package can
% be obtained from http://gaia.fdi.ucm.es/projects/texis/
%
% Although the TeXiS template itself is distributed under the
% conditions of the LaTeX Project Public License
% (http://www.latex-project.org/lppl.txt), the manual content
% uses the CC-BY-SA license that stays that you are free:
%
%    - to share & to copy, distribute and transmit the work
%    - to remix and to adapt the work
%
% under the following conditions:
%
%    - Attribution: you must attribute the work in the manner
%      specified by the author or licensor (but not in any way that
%      suggests that they endorse you or your use of the work).
%    - Share Alike: if you alter, transform, or build upon this
%      work, you may distribute the resulting work only under the
%      same, similar or a compatible license.
%
% The complete license is available in
% http://creativecommons.org/licenses/by-sa/3.0/legalcode
%
%---------------------------------------------------------------------
%
% Contiene la pígina de agradecimientos.
%
% Se crea como un capítulo sin numeraciín.
%
%---------------------------------------------------------------------

\chapter{Agradecimientos}
\cabeceraEspecial{Agradecimientos}

\begin{FraseCelebre}
\begin{Frase}
I find your lack of faith disturbing\end{Frase}
\begin{Fuente}
Darth Vader, Star Wars: A New Hope.
\end{Fuente}
\end{FraseCelebre}

Me gustaría agradecer a los profesores Samer Hassan collado y Antonio Sanchez Ruiz-Granados
por el apoyo prestato para realizar este trabajo. 
\\
\mbox{}
\\
Al departamento de ISIA por fomentar mis ideas y pontenciarlas.
\\
\mbox{}
\\
A la universidad complutense por brindarme el conocimiento necesario para poder
realizar las ideas necesarias.
\\
\mbox{}
\\
A todos los que piensan que este trabajo no llegará a ningun lado, que me dan
fuerzas para seguir adelante.
\endinput
% Variable local para emacs, para  que encuentre el fichero maestro de
% compilaciín y funcionen mejor algunas teclas rípidas de AucTeX
%%%
%%% Local Variables:
%%% mode: latex
%%% TeX-master: "../Tesis.tex"
%%% End:


%---------------------------------------------------------------------
%
%                      resumen.tex
%
%---------------------------------------------------------------------
%
% Contiene el cap_tulo del resumen.
%
% Se crea como un cap_tulo sin numeraci_n.
%
%---------------------------------------------------------------------

\chapter{Abstract}
\cabeceraEspecial{Abstract}

\begin{FraseCelebre}
\begin{Frase}
...
\end{Frase}
\begin{Fuente}
...
\end{Fuente}
\end{FraseCelebre}

... Resumen chulo

\endinput
% Variable local para emacs, para  que encuentre el fichero maestro de
% compilaci_n y funcionen mejor algunas teclas r_pidas de AucTeX
%%%
%%% Local Variables:
%%% mode: latex
%%% TeX-master: "../Tesis.tex"
%%% End:


\ifx\generatoc\undefined
\else
%---------------------------------------------------------------------
%
%                          TeXiS_toc.tex
%
%---------------------------------------------------------------------
%
% TeXiS_toc.tex
% Copyright 2009 Marco Antonio Gomez-Martin, Pedro Pablo Gomez-Martin
%
% This file belongs to TeXiS, a LaTeX template for writting
% Thesis and other documents. The complete last TeXiS package can
% be obtained from http://gaia.fdi.ucm.es/projects/texis/
%
% This work may be distributed and/or modified under the
% conditions of the LaTeX Project Public License, either version 1.3
% of this license or (at your option) any later version.
% The latest version of this license is in
%   http://www.latex-project.org/lppl.txt
% and version 1.3 or later is part of all distributions of LaTeX
% version 2005/12/01 or later.
%
% This work has the LPPL maintenance status `maintained'.
%
% The Current Maintainers of this work are Marco Antonio Gomez-Martin
% and Pedro Pablo Gomez-Martin
%
%---------------------------------------------------------------------
%
% Contiene  los  comandos  para  generar los  �ndices  del  documento,
% entendiendo por �ndices las tablas de contenidos.
%
% Genera  el  �ndice normal  ("tabla  de  contenidos"),  el �ndice  de
% figuras y el de tablas. Tambi�n  crea "marcadores" en el caso de que
% se est� compilando con pdflatex para que aparezcan en el PDF.
%
%---------------------------------------------------------------------


% Primero un poquito de configuraci�n...


% Pedimos que inserte todos los ep�grafes hasta el nivel \subsection en
% la tabla de contenidos.
\setcounter{tocdepth}{2}

% Le  pedimos  que nos  numere  todos  los  ep�grafes hasta  el  nivel
% \subsubsection en el cuerpo del documento.
\setcounter{secnumdepth}{3}


% Creamos los diferentes �ndices.

% Lo primero un  poco de trabajo en los marcadores  del PDF. No quiero
% que  salga una  entrada  por cada  �ndice  a nivel  0...  si no  que
% aparezca un marcador "�ndices", que  tenga dentro los otros tipos de
% �ndices.  Total, que creamos el marcador "�ndices".
% Antes de  la creaci�n  de los �ndices,  se a�aden los  marcadores de
% nivel 1.

\ifpdf
   \pdfbookmark{Índices}{indices}
\fi

% Tabla de contenidos.
%
% La  inclusi�n  de '\tableofcontents'  significa  que  en la  primera
% pasada  de  LaTeX  se  crea   un  fichero  con  extensi�n  .toc  con
% informaci�n sobre la tabla de contenidos (es conceptualmente similar
% al  .bbl de  BibTeX, creo).  En la  segunda ejecuci�n  de  LaTeX ese
% documento se utiliza para  generar la verdadera p�gina de contenidos
% usando la  informaci�n sobre los  cap�tulos y dem�s guardadas  en el
% .toc
\ifpdf
   \pdfbookmark[1]{Tabla de contenidos}{tabla de contenidos}
\fi

\cabeceraEspecial{\'Indice}

\tableofcontents

\newpage

% �ndice de figuras
%
% La idea es semejante que para  el .toc del �ndice, pero ahora se usa
% extensi�n .lof (List Of Figures) con la informaci�n de las figuras.

\cabeceraEspecial{\'Indice de figuras}

\ifpdf
   \pdfbookmark[1]{Índice de figuras}{indice de figuras}
\fi

\listoffigures

\newpage

% �ndice de tablas
% Como antes, pero ahora .lot (List Of Tables)

\ifpdf
   \pdfbookmark[1]{Índice de tablas}{indice de tablas}
\fi

\cabeceraEspecial{\'Indice de tablas}

\listoftables

\newpage

% Variable local para emacs, para  que encuentre el fichero maestro de
% compilaci�n y funcionen mejor algunas teclas r�pidas de AucTeX

%%%
%%% Local Variables:
%%% mode: latex
%%% TeX-master: "../Tesis.tex"
%%% End:

\fi

% Marcamos el  comienzo de  los cap_tulos (para  la numeraci_n  de las
% p_ginas) y ponemos la cabecera normal
\mainmatter
\restauraCabecera

%\include{Capitulos/Parte1}
%\include{Capitulos/Abstract}

\chapter{Introduction, objectives and document structure}
\label{intro}
\begin{FraseCelebre}
  \begin{Frase}
    Who Watches the Watchmen?
  \end{Frase}
  \begin{Fuente}
    Watchmen - Alan Moore
  \end{Fuente}
\end{FraseCelebre}

\section{Introduction}

Scientific research nowadays is based on publishing in journals with a high
\bb{impact factor} (\emph{IF})\cite{doi:10.1001/jama.295.1.90}, a researcher's
career can be measured depending on the number of papers published in these
journals. There are different impact factors that determine the quality of a
journal. One of the most well-known is the Journal Citation Reports
(\emph{JCR}), an indicator that represents, for each indexed journal, the
relation between the number of citable items and the number of citations they
get. This \emph{IF} is calculated every year and it is divided in four quartiles
that determine the raking of a certain journal, meaning that a journal in the
first quartile (Q1) has higher impact factor than one in the second(Q2), third
(Q3) or fourth(Q4) quartiles\footnote{https://jcr.incites.thomsonreuters.com/}.
\emph{JCR} was originally an evolution of the Science Citation Index, born in
1955~\cite{garfield2007evolution} and nowadays managed by a company called
Thomson Reuters\footnote{https://www.thomsonreuters.com}. \emph{ Could the
  scientific community rely upon themselves rather than on a private company to
  decide the quality of an academic journal?}

One of the problems in academia is the publishing obsession. Ideally, a research
must achieve publications in indexed journals. This idea causes, in some cases,
to yield to the demands of the reviewers or the editors of a journal,
potentially reducing the originality or novelty of a research
paper~\cite{Frey2003}. In some fields such as marketing, universities are
increasingly pressuring researchers to publish papers in journals with
high-quality journal impact metrics, forcing them to focus their research
projects on generating publishable material~\cite{ortinau2011writing}. \emph{ Is
  there a better way to make a career in academia rather than forcing
  researchers to generate redundant or unoriginal publications?}

Science publication and peer review are build on a paper-based paradigm, with
only a few changes in the last centuries~\cite{spier2002history}. The mentioned
\emph{peer review} is the process to decide if a paper is suitable to be
published. A group of ``experts'' in a certain subject review the paper and
issue a verdict based on whether or not it is published. But this process has
been criticized in several aspects. The reviews are not always entirely
objective, since there are cases of unfavorable reviews due to gender causes,
especially in scientific fields~\cite{wenneras2001nepotism}. The review time of
a paper is usually long, causing the process of academic research to be quite
slow~\cite{huisman2017duration}. The reviews not always ensures the quality of a
paper, and cannot be used to decide if a research is good or
not~\cite{goldbeck1999evidence}. \emph{Would it be possible to find better
  alternatives to improve this process so that it is more honest, fair or fast?}

The benefits of scientific distribution are centralized in a few publishers, nor
the authors, the reviewers or the readers get money from it. The development of
the Internet enabled the proposal of alternatives for science
dissemination~\cite{eysenbach2006citation} and
evaluation~\cite{walker_emerging_2015}. The reduction of distribution costs
enabled a wider access to scientific knowledge, and questioned the role of
traditional publishers~\cite{ReinventingRigor}. Nevertheless, universities
normally have to take charge of the costs to access the papers published in
these journals, paying unfair fees~\cite{bergstrom2004costs}. On the other hand,
Open Access and Open Science movements have successfully reduced the economic
cost of accessing knowledge to readers~\cite{evans2009open}. However, it has not
successfully challenged traditional publishers' business
models~\cite{lariviere2015oligopoly}, who are now combining charging readers and
charging authors~\cite{van2013true}. \emph{Could the scientific community build
  a system to decentralize the benefits of science publication and reward the
  authors and the reviewers for their work?}

Editors who want to assign the review of a paper to a series of reviewers have
to rely on them beforehand. Thus, limiting the spectrum of fields that can be
reviewed to the fields in which those reviewers are experts. The internet offers
the possibility to meet people all around the world, and when it comes to trust
total strangers, it should be a system in which anyone can rely to find trustful
people. Reputation systems are the solution to these problems, since they offer
a good first impression about an unknown person~\cite{resnick2000reputation}. If
an editor want to broaden the scope of reviewers with more fields of expertise,
she will need to contact new reviewers. But there is no easy way to predict
reviewer quality from their training and experience factors
\cite{callaham_relationship_2007}. \emph{Could there be a system in which
  reviewers get rated based on their reviews and build up their reputation based
  on good practices and helpful reviews?}

Peer review has suffered multiple criticism, and yet only marginal alternatives
have gathered success~\cite{ware2008peer}. The literature provides multiple
proposals around open peer review~\cite{ford2013defining}, and proposals of
reputation networks for reviewers~\cite{frishauf2009reputation}. In fact, a
start-up, Publons\footnote{https://publons.com/}, provides a platform to
acknowledge reviews and open them up. To sum up, this work aims to solve many of
the problems mentioned before trying to answer the questions asked previously.

\section{Objectives}
\label{sec:objectives}
\subsection*{Create a decentralized platform for science publishing}

This work proposes the design and development of a platform for open science.
This platform should allow its users to do the following interactions: submit
papers, assign reviewers, submit reviews and rate the reviews. This will be
achieved using decentralized technologies such as Ethereum (a decentralized
ledger where each interaction is recorded in a public blockchain) and IPFS (a
distributed file system).

The platform should be accessible through a web page in a custom server acting
as a bridge between the two technologies mentioned before. Also should give its
users the possibility to download the source code and running it locally, as it
is a fully decentralized platform.

All the information held by the platform will be free and public, granting the
possibility to each user to see all the journals, papers, reviews and ratings.

\subsection*{Create a reputation system for reviewers}

Reviewers rarely get credit for their work. Journals and conferences look for
\emph{volunteers} to review the papers submitted to those, but normally
reviewers remain anonymous. To
achieve this recognition, this work proposes a reputation system for reviewers,
in which every review they submit can be rated. This rating builds up a score
for each reviewer determining the \emph{reputation} of a reviewer.

The platform should allow users to send ratings to an specific review. This
rating will be saved and will be used to calculate the score of the reviewer
that sent that particular review.

With this idea, this work pretends to \emph{foment} good and fair reviews, and
avoid bad ones, trying to mitigate the possibility of unfair reviews due to
gender causes, research rivalry or ignorance of a subject.

\subsection*{Analyze the platform}

After developing a functional prototype with which to perform tests, this work
also aims to analyze the behavior of the platform based on the entire process of
scientific publication, from the first paper submission to the final
publication.

Tests will be carried out to calculate price estimates of the entire process,
execution times, resistance to large amounts of information and monetary impact
within the academic community.

Finally, comparisons of the results obtained will be shown based on the current
publication process.

%%%%%%%%%%%%%%%%%%%%%%%%%%%%%%%%%%%%%%%%%%%%%%%%%%%%%%%%%%%%%%%%%%%%%%%%%%%%%%%%%%%%
% This work aims to challenge middlemen such as traditional publishers in
% science %
% publication. Particularly, proposes a decentralized publication system for
% open %
% science, allowing 1) paper submissions, 2) assignment of reviewers, 3) peer %
% review and, as a novelty, 4) the rating of peer reviews. With this
% distributed %
% system, we aim to improve the quality and efficiency of reviews and
% knowledge %
% distribution, helping editors, authors, and reviewers: %
% \begin{itemize} %
% \item Editors and journals will be able to find the best peer reviewers in
%   their %
%   fields of interest, and also those that respond quickly. Thus, reducing %
%   time-to-publish and publishing costs. %
% \item Authors will be able to submit papers to time-responsive, free, open %
%   access journals, and forget about slow, unfair and unaccountable anonymous %
%   reviews. %
% \item Reviewers will finally have their work recognized. %
% \end{itemize} %
%                                                                                  %
% We are interested in exploring the following challenges, that could be
%                                                                                  % dealt %
% with our technology: %
% \begin{itemize} %
% \item Reduce time-to-review by rewarding on-time reviewers. %
% \item Measure and prevent sexism, nepotism and other abuses in peer review. %
% \item Develop fully autonomous decentralized journals. %
% \item Explore fully free publication systems for Open Access science, while %
%   enabling innovative business models. %
% \item Explore alternative and open metrics for papers, journals and
%   reviewers. %
% \end{itemize} %
%%%%%%%%%%%%%%%%%%%%%%%%%%%%%%%%%%%%%%%%%%%%%%%%%%%%%%%%%%%%%%%%%%%%%%%%%%%%%%%%%%%%

\section{Document Structure}
In this work there will be the following sections:

\subsubsection*{Part 1: Preface.}
\begin{itemize}
  \itbf{Background and State of the art:} Background about the scope of the
  project and what technologies are trying to change the current publication
  systems and how is affecting the scientific community.

  \itbf{Methodology and Technology:}Methodology followed during the realization
  of this work, and the technologies used to implement the platform's
  architecture.
\end{itemize}
\subsubsection*{Part 2: Decentralized Science.}
\begin{itemize}
  \itbf{Platform description:} Platform general description, featuring its main
  strengths regarding the current platforms and explaining how it works and what
  is the expected behavior if it is widely used in the future.

  \itbf{Architecture:} Technical description of the platform, including the
  front end architecture and the definition of the smart contracts'
  infrastructure, and the process followed to reduce the interaction costs.

  \itbf{Product:} Proof of concept of the platform, how it works and how the
  users can interact with the blockchain and with the p2p file network.

  \itbf{Discussion:} Results obtained after the realization of the work proposed
  in this project, how it will affect the scientific community and how to
  measure the potential impact of the platform.

  \itbf{Conclusion and future work:} Implications of this work in the scientific
  community and the next steps to follow to create an ecosystem of autonomous
  publication systems, without the need of middlemen such as journals or
  editors, and a proposal of a future Ph.D. about this subject.
\end{itemize}

\cleardoublepage

\section{Introducción}

La investigación científica hoy en día se basa en publicar en revistas con alto
\bb{índice de impacto} \cite{doi:10.1001/jama.295.1.90}, la carrera de un
investigador se puede medir en función del número de artículos académicos que ha
publicado en estas revistas. Hay diferentes índices de impacto que determinan
la calidad de una revista. Uno de los más conocidos es el \ii{Journal Citation
  Reports} (JCR), un indicador que representa, para cada revista indexada, la
relación entre el número de objetos citables y en número de citas que obtienen.
Este factor es calculado cada año y está dividido en cuatro cuartiles que
determinan el ranking de cada revista. Esto quiere decir que una que esté
en el primer cuartil (Q1) tiene mayor índice de impacto que las de los
cuartiles dos (Q2), tres (Q3) o cuatro
(Q4)\footnote{https://jcr.incites.thomsonreuters.com/}. \ii{JCR} fué
originalmente una evolución de el llamado \emph{Science Citation Index}, que
nació en 1955~\cite{garfield2007evolution} y que hoy en día es controlado por
una compañía privada llamada ``Thomson
Reuters''\footnote{https://www.thomsonreuters.com}. \ii{¿Podría la comunidad
  científica depender de si misma en vez de en una compañía privada para decidir
  la calidad de una revista académica?}

Uno de los problemas en el mundo de la academia es la obsesión por publicar.
Idealmente, una investigación tiene que conseguir muchas publicaciones en
revistas indexadas. Esta idea causa, en algunos casos, que los autores de los
artículos tengan que suplir las exigencias que les imponen los revisores y los
editores de estas revistas, reduciendo potencialmente la originalidad y la
novedad de un artículo de investigación~\cite{Frey2003}. Uno de los ejemplos más
claros es en el campo de investigación de mámarketing, en el que las
universidades presionan cada vez más a los investigadores para publicar en
revistas de alto impacto, forzando a estos a centrar su investigación en generar
material publicable~\cite{ortinau2011writing}. \emph{ ¿Podría existir una forma
  mejor de hacer carrera en el mundo de la academia en vez de forzar a los
  investigadores a generar material redundante o poco original?}

La publicación científica y el proceso de revisión por pares están construidos
sobre un paradigma basado en artículos, con pocos cambos en los últmos
siglos~\cite{spier2002history}. El mencionado proceso de \ii{revisión por pares}
es el que se utiliza hoy en día para decidir si un artículo académico es
apto para ser publicado o no. Un grupo de ``expertos'' en una materia en
concreto revisan el artículo y emiten un veredicto.
Pero este proceso ha sido criticado en varios aspectos tales como: 1) Las revisiones no
siempre son del todo objetivas, ya que existen casos de revisiones desfavorables
por causas de género, especialmente en los campos de ciencias
puras~\cite{wenneras2001nepotism}. 2) El tiempo de revisión de un artículo suele
ser largo, provocando que el proceso de investigación en algunos casos se
ralentice~\cite{huisman2017duration}. 3) Las revisiones no siempre garantizan la
calidad de un artículo académico, y no pueden ser utilizadas para decidir si una
investigación es buena o no~\cite{goldbeck1999evidence}. \emph{¿Podrían
  explorarse alternativas para mejorar este proceso para que sea más honesto,
  justo o rápido?}

Los beneficios de la distribución científica están centralizados en unas pocas
editoriales, ni los autores, los revisores o los lectores obtienen dinero de
este sistema. Además, a pesar de que el desarrollo y la expansión de Internet
han proporcionado nuevas formas de diseminación~\cite{eysenbach2006citation} y
evaluación ~\cite{walker_emerging_2015} para el proceso de publicación, los
beneficios siguen concentrados en dichas editoriales. La reducción de costes de
distribución ha causado un mayor acceso al conocimiento científico, y por ello
se ha cuestionado el papel de las editoriales tradicionales en este
sistema~\cite{ReinventingRigor}. Sin embargo, las universidades normalmente
asumen los costes de acceso a los artículos publicados en estas revistas a
través de subscripciones algunas veces injustas~\cite{bergstrom2004costs}. Por otra parte, los
movimientos \ii{Open Access} y \ii{Open Science} han reducido con éxito estos
costes para los lectores~\cite{evans2009open}. No obstante, esto no ha sido
suficiente para paliar los beneficios del modelo de negocio de la editoriales
tradicionales~\cite{lariviere2015oligopoly}, las cuales cobran a los autores en
vez de a los lectores~\cite{van2013true}. \emph{¿Podría la comunidad científica
  construir un sistema para descentralizar los beneficios del proceso de
  publicación y recompensar tanto a los autores como a los revisores?}

Los editores de una revista que nececesiten asignar a los revisores para los
artículos que reciben tienen que confiar en ellos de antemano. Esto puede
limitar el espectro de campos para los que un editor pueda encontrar revisores.
Pese a ello, internet ofrece la posibilidad de encontrar a gente a través de
todo el mundo, pudiendo encontrar revisores expertos en todo tipo de materias.
Sin embargo, cuando se trata de confiar en una persona desconocida, debería
existir un sistema en el que cualquiera pueda confiar para encontrar revisores
de calidad. Los sistemas de reputación son la solución a este problema, ya
que ofrecen una primera impresión de una persona basada en las opiniones de
otras que hayan interactuado con ella~\cite{resnick2000reputation}. Si un editor
de una revista quiere ampliar su plantilla de revisores necesitaría contactar
con nuevas personas, las cuales pueden ser desconocidas. El problema es que no
es fácil determinar la calidad de un revisor basada en su experiencia
\cite{callaham_relationship_2007} por lo que un sistema de reputación podría ser
útil para encontrar buenos revisores sin tener que conocerlos de antemano.
\ii{¿Se podría construir un sistema en el que los revisores son puntuados para
  conseguir reputación basada en lo buenas o malas que sean sus revisiones?}

\section{Objetivos}
\label{sec:objectives-1}
\subsection*{Objetivo 1: Crear una plataforma descentralizada para la publicación
  científica}

Este trabajo propone el diseño y el desarrollo de una plataforma para \ii{Open
  Science}. Esta plataforma debería permitir a los usuarios: enviar artículos,
asignar revisores, enviar revisiones y puntuar estas revisiones. Para ello se
utilizarán tecnologías descentralizadas como Ethereum (una plataforma
distribuida en la que cada interacción es grabada en una base de datos pública)
e IPFS (un sistema de archivos distribuido).

La plataforma será accesible a través de una web en un servidor personalizado
que actuará como puente entre las dos tecnologías mencionadas. Además, al ser
distribuido, se ofrecerá a los usuarios el código fuente para poder ejecutar esta en un nodo local.

Toda la información de la plataforma será pública y gratuita, dando la
posibilidad de obervar las revistas, los artículos académicos, las revisiones
y la reputación de todos los revisores.

\sst{Objetivo 2: Crear un sistema de reputacion para revisores}

Los revisores raras veces obtienen reconocimiento por su trabajo. Las revistas
científicas y las conferencias normalmente buscan \ii{voluntarios} para llevar
a cabo el proceso de revisión, pero en la mayoría de los casos, los revisores permanecen anónimos. Para
conseguir este reconocimiento, este trabajo propone el desarrollo de un sistema
de reputación de revisores, en el que cada revisión puede ser puntuada. Esta
puntiación es asociada a cada revisor, la cual determina la \ii{reputación} que
tiene.

La plataforma debe permitir a los usuarios calificar cada una de las revisiones
que obtienen los artículos académicos. Esta calificación servira para calcular
la puntuación de cada revisor dentro de la plataforma, haciendo que los
revisores que realicen buenas revisiones tenga buena reputación y los que no
mala.

Con esta idea este trabajo pretende \ii{fomentar} revisiones buenas y entregadas
a tiempo y disuadir a aquellos revisores que no estén dispuestos a esto,
mitigando en la medida de lo posible revisiones injustas debido a causas de
genero, rivalidad de investigación o desconocimiento de una materia.

\sst{Objetivo 3: Analizar la plataforma}

Después de desarrollar un prototipo funcional para realizar pruebas, este
trabajo propone analizar el comportamiento de la plataforma basado en el proceso
completo, desde el envío del primer artículo hasta su publicación
final.

Se realizarán varios  test para calcular estimaciones del coste de todo el
proceso, tiempos de ejecución, resistencia a grandes cantidades de información e
impacto monetario en la comunidad.

Finalmente se ofrecerán conclusiones de los resultados obtenidos para analizar
la viabilidad de la implantación de esta plataforma frente a los sistemas
actuales.


\section{Estructura del documento}
Este trabajo dispone de los siguientes capítulos:


\subsubsection*{Part 1: Preface.}
\begin{itemize}
  \itbf{Background and State of the art:} Trasfondo en
  el cual incide el proyecto y diferentes tecnologías actuales para
  cambiar el proceso actual de publiación de ciencia.

  \itbf{Methodology and Technology:} Explicación de  las
  metodologías y tecnologías utilizadas durante el desarrollo de todo el
  proyecto.
  
\end{itemize}
\subsubsection*{Part 2: Decentralized Science.}
\begin{itemize}
  \itbf{Platform description:} Descripción general de la plataforma, presentando
  el funcionamiento esperado y las principales ventajas del diseño.
  
  \itbf{Architecture:} Descripción técnica de la plataforma, incluyendo la
  arquitectura del \ii{frontend}, la definición del funcionamiento interno y el
  proceso seguido para reducir los costes de interacción.
  
  \itbf{Product:} Prueba de concepto de la plataforma, mostrando un prototipo
  funcional que interactúa con las tecnologías distribuidas mencionadas previamente.

  \itbf{Discussion:} Resultados obtenidos después del desarrollo de la
  plataforma y cómo estos afectarían a la comunidad científica.

  \itbf{Conclusion and future work:} Implicaciones de este trabajo,
  observaciones finales, y propuestas para un futuro proyecto de doctorado.
\end{itemize}

%%%
%%% Local Variables:
%%% mode: latex
%%% TeX-master: "../Tesis.tex"
%%% End:


%%%
%%% Local Variables:
%%% mode: latex
%%% TeX-master: "../Tesis.tex"
%%% End:

\chapter{State of the art}

\begin{FraseCelebre}
  \begin{Frase}
    The needs of the many outweigh the needs of the few
  \end{Frase}
  \begin{Fuente}
    Spock - The Wrath of Khan
  \end{Fuente}
\end{FraseCelebre}

% -------------------------------------------------------------------
% \section{Cooler Section}
% -------------------------------------------------------------------



\section{Alternative Publication systems}

Publication systems, as seen on section \ref{intro} are vampirizing the
industry, nevertheless there are some attempts to change this paradigm on behalf
of science dissemination.

Open journal systems~\cite{willinsky2005open} is an open software designed to
facilitate the publishing process. This project was created by the Public
Knowledge Project\footnote{https://pkp.sfu.ca/about/} and it targets open-access
online journals that want to speed up the publication processes. The system
provides tools to control the whole publishing process from article submission,
through peer reviewing to the final publication issue.

Mega-journals~(or Multi-journals)~\cite{binfield2013open,wellen2013open} combine
multiple journals into a single journal, allowing the publication of open-access
papers, which have gone through a peer review process. The first journal to
adopt this idea is the \emph{PLOS ONE}
Journal\footnote{http://journals.plos.org/plosone/} as of the project
\emph{Public Library of Science}. Project that aims to create a library of
scientific journals under the values of open access and creative commons
licenses. As a result of the success of the \emph{PLOS ONE} journal, other
publishers have started their own mega-journals. Featuring alternative impact
metrics, reusability of figures and data, post-publication discussions and
portable reviews from other journals~\cite{bjork2015have}.

Continuous publication model is based on publishing individual papers migrating
from the previous issue-based model~\cite{anderton2013continuous}. This method
is seen as an altenative for open-access journals as it speeds up the
publication process~\cite{haymanview}. \emph{DPSOS}\footnote{Decentralized
  Publication System for Open Science} adopt this model by design (see
section~\ref{tech:sec:ethereum:sm}) as it publish automatically the papers that
meet certain preconditions that are written in the blockchain.

Preprints are scientific papers that have not yet gone through the peer review
process~\cite{harnad2003electronic}. Formerly, the preprints that were sent to
the journals were private, and only accessible by the editors and assigned
reviewers. But nowadays it is common to publish a preprint before sending it to
a journal, uploading it to specialized platforms like arXiv
\footnote{https://arxiv.org/} or Preprints
\footnote{https://www.preprints.org/}~\cite{brown2001volution}. In fact there is
a correlation bet@incollection{resnick2002trust, title={Trust among strangers in
    Internet transactions: Empirical analysis of eBay's reputation system},
  author={Resnick, Paul and Zeckhauser, Richard}, booktitle={The Economics of
    the Internet and E-commerce}, pages={127--157}, year={2002},
  publisher={Emerald Group Publishing Limited} }ween the upload of a preprint
and the early citations after the publication of the
paper~\cite{shuai2012scientific}. This system is a possible solution to the
cold-start problem that have the papers of new researchers who enter the
academic career~\cite{sugiyama2010scholarly}.

Social networks have also made a dent in the academic world, creating platforms
to contact other researchers and encouraging them to share open access papers.
Some of the well known are Research Gate\footnote{https://www.researchgate.net},
Mendeley \footnote{https://www.mendeley.com} or Academia
\footnote{http://academia.edu}. But despite the good intentions of the creators
of these platforms, many of the journals demand the copyright of the papers they
publish, preventing the authors from sharing them through these services.


Decentralized alternatives, despite their promises~\cite{bartlingblockchain},
are still in their infancy. A few proposals, none of them functional to date,
have appeared recently.

A peer review proposal to solve some of the peer review socio-technical problems
using cryptocurrencies~\cite{tennant2017multi}. It needs a critical threshold of
research community engagement, changing the actual processes and platforms, to
start being implemented.

Blockchain-enabled apps have also been proposed, with voting and storage of
publications. This is the case of Aletheia~\cite{morton2017aletheia}, a software
for getting open access papers published. This platform idea aims to use
blockchain as a decentralized and distributed database as a publishing platform.

Peer review quality control through blockchain-based cohort
trainings~\cite{dhillon2016bench} have been also proposed, with the promise of
transparency and dezentralization using a distributed ledger. Research labs can
use this training network to test their technology and reduce the risk for
private investment opportunities.

Finally, some of the off-chain journals are adapting to the demands of the
current scientific community like Ledger\footnote{https://ledgerjournal.org}, a
cryptocurrencies and blockchain based journal that records the publication
timestamps in the Bitcoin blockchain.

\section{Reputation systems}

Reputation systems today arise from the need to trust unknown
individuals~\cite{resnick2000reputation}, many of the big internet communities
like Stackexchange\footnote{https://stackexchange.com/} or
reddit\footnote{https://www.reddit.com/} have their own reputation system.
Reputation systems behavior may vary depending on the
platform~\cite{josang2002beta}, but the most usual is where users get a score
based on certain interaction with the community.

Reputation systems also have a very large niche in e-commerce webs such as
Ebay\footnote{https://www.ebay.com}, in which people pay for a product sold by
an unknown vendor. There must be a previous trust in the vendor before buying
any product, a reputation system offers an score given by other users to trust
or not that certain seller~\cite{resnick2002trust}.

Reputation systems vary widely in scope, such as one for peer-to-peer
computing~\cite{zhou2007powertrust}, vehicle ad-hoc~\cite{dotzer2005vars} and
even Wikipedia~\cite{adler2007content}. And all of them are based on an exchange
of trust between users who use these services.

This same concept was intended to be transferred within the blokchain using a
token as a trust unit, in which users exchanged these as a sign of trust
deposits among them~\cite{sharples2016blockchain}.


This paper proposes the development of a decentralized publication system for
open science. It aims to challenge the technical infrastructure that supports
the middlemen role of traditional publishers. Due to the successes of the Opene
Access movement, some of the scientific knowledge is today freely provided by
the publishers. However, the content is still mostly served from their
infrastructure (i.e. servers, web platforms). This ownership of the
infrastructure gives them a power position over the scientific community which
produces the contents~\cite{fuster2010governance}. Such central and
oligopolistic position in science dissemination allows them to impose policies
(e.g. copyright ownership, Open Access prices) and concentrate profits.

The proposed system aims to move the infrastructure control from the publishers
to the scientific community. It entails the decentralization of three essential
functions of science dissemination: 1) the peer review process, 2) the selection
and recognition of peer reviewers, and 3) the distribution of scientific
knowledge. The following section provides an overview of the system features,
while the final section discusses its challenges.
%%%
%%% Local Variables:
%%% mode: latex
%%% TeX-master: "../Tesis.tex"
%%% End:

\chapter{Methodology \& Technology}

\begin{FraseCelebre}
  \begin{Frase}
    The needs of the many outweigh the needs of the few
  \end{Frase}
  \begin{Fuente}
    Spock - The Wrath of Khan
  \end{Fuente}
\end{FraseCelebre}

\section{Methodology}
The idea of this project came up in a Hackathon in September 2017. We were a
group of 4 developers with one month to create an idea and a small prototype to
implement using blockchain technologies.

To give birth to the idea of a decentralized publication system for open science
we used agile methodologies.


\subsection{Brainstorming}

Brainstorming was born as a method to increase creativity in groups and
organizations. There are only few rules on this method: do not criticize any of
the given ideas, quantity is desired over quality, try to combine suggested
ideas and give all the ideas that come to mind, no matter if they are possible
or not~\cite{osborn1953applied}.

This method is used nowadays in companies and work groups as part of the process
of the creatfile uploadsion of a product, although there are some critics about
brainstorming and sometimes instead of encouragind creativity, inhibits
it~\cite{sutton1996brainstorming,mullen1991productivity}

Leaving apart these problems, we decided to make a brainstorming session to
define what we were going to do. Many ideas emerged and were capture into a
white board without discrimination, no matter how hard or easy to implement they
were.

After saying enough ideas to fill the board we filtered the ones that were
impossible to achieve. Then, each one voted the best three, making a ranking of
the 3-4 best projects to start working on. Some of the ideas were creating a
distributed \emp{wikipedia} with governance models, an application to contact
people from minority groups in countries where they are persecuted collectives,
a distributed and community driven NGO, and a crowdfunding platform for
\emph{whisteblowers}.

Finally we decided to create an approach to a distributed platform for open
science.

\subsection{Value proposition canvas}

\figura{vpc.jpg}{width=0.9\linewidth}{vpcimage}{Image of the value proposition
  canvas after the session}

A value proposition canvas is a tool to create, design and implement a product
idea. Is commonly used by businesses and entrepreneurs to find the balance
between customer profile and product design, but there are other cases of use
for this tool outside business
scope~\cite{pokorna2015value,meertens2012mapping}.

The process is divided in two parts, customer profile and value map, each of
these divided in other three parts:~\cite{osterwalder2014value}:

\begin{itemize}
\item \textbf{Customer profile:} This step is to identify the profile of the
  final user of the platform. This section is divided in three parts: 1)
  \emph{Customer jobs:} things the customer are trying to get done, 2)
  \emph{Customer pains:} undesired costs and situations, 3) \emph{Customer
    gains:} benefits, social gains and cost savings expected.
\item \textbf{Value Map:} This section is about what the final product has to
  have and what does not, and its also divided in: 1) \emph{Product and
    services:} which products and services are offered that help the customer
  get a job done, 2) \emph{Pain relievers:} how the customer pains are going to
  be alleviated, 3) \emph{Gain creators:} how the products and services create
  customer gains
\end{itemize}

We decided to use this methodology for the definition of the final platform,
since it established the general development framework of the application.

\subsection{Agile methodologies} \TODO{REPASAR}

El gran crecimiento de internet y la economía digital ha alterado el concepto de
ingeniería del software. Las metodologías de desarrollo de software (\emph{MDS})
tradicionales están siendo eclipsadas por nuevas \emph{MDS} ligeras o \emph{MDS}
ágiles. Estas metodologías (en adelante \emph{MDSAs}) están caracterizadas por
la integración continua, desarrollo iterativo y la capacidad de asumir cambios
en los requerimientos de negocio.\cite{boehm2005management,livermore2008factors}

La \emph{MDSAs} más popular es conocida como programación extrema o
\emph{Extreme Programming (XP)}\cite{lindstrom2004extreme} basada en una serie
de conceptos básicos a la hora de realizar el desarrollo de un programa:
simplicidad del código y prototipado rápido, comunicación continua del cliente
con el equipo de desarrollo, responsabilidad del código de todos los integrantes
del grupo, reuniones cortas y rápidas, refactorización e integracción
continua.\cite{theunissen2005search,livermore2008factors}

Otros métodos dentro de las \emph{MDSAs} son scrum\cite{rising2000scrum},
métodos cristalinos\cite{cockburn2004crystal} y desarrollo basado en
funcionalidades (en inglés \emph{FDD})\cite{coad1999java}. El uso de estas
\emph{MDSAs} permite a los desarrolladores crear software de mejor calidad en
periodos de tiempo más cortos. Estas metodologías han sido desarrolladas para
mejorar el proceso de desarrollo, quitando las barreras a la aceptación de
cambios en los requerimientos del cliente, un hecho que se da con bastante
frecuencia\cite{lindstrom2004extreme}.

\section{Technology}
\label{tech}
To face the challenges proposed by this project, there are many possibilities to
distribute both the data and the information about the reputation network. To
build a robust and decentralized system i

% -------------------------------------------------------------------
\subsection{IPFS}
% -------------------------------------------------------------------
\label{tech:sec:ipfs}
IPFS stands for Interplanetary File System. It is a peer-to-peer file-sharing
protocol that uses a cryptographic hashes to store files in a distributed
network. IPFS works very similar to HTTP protocol but in a BitTorrent way. It
can be seen as a giant git repository where everyone can store, share and
exchange files\cite{benet2014ipfs}.

IPFS merges three main ideas: Distributed Hash Tables, BitTorrent, Git and
Self-Certified Systems.

\subsubsection{Distributed Hash Tables}
\label{tech:sec:ipfs:dht}
A distributed hash table(\emph{DHT}) is a decentralized structure that works
very similar to a hash table. Hash tables are used to identify items in a
database. The table performs simple mathematical operations generating a random
string called hash. The hash acts as a pointer that directs to the data, this
allows the user to find data directly instead of looking through the entire
database\cite{kaluszka2010distributed}.

In a distributed hash table, any node can use a hash as a key to retrieve data.
This system includes a data structure called ``keyspace'' that is a set of all
possible keys, which is split up across the nodes in the system. The mapping of
the keys is made by another function that describes the distance from one key to
another. All the nodes have and identifier and a set of identifiers pointing to
all its neighbors nodes. If a node is removed from the network, only a small
portion of the data must be recovered by other
nodes\cite{kaluszka2010distributed}.

This system makes \emph{DHTs} scalable, fast and robust. It is used by
frameworks such as Tapestry \cite{zhao2004tapestry}, Chord
\cite{stoica2001chord}, Kelips \cite{gupta2003kelips}, Kademlia
\cite{maymounkov2002kademlia} and IPFS \cite{benet2014ipfs}. These platforms are
similar in cost and performance if they are tested in a large enough network.
They behave very fast when it comes to searching for a key through massive
networks of nodes\cite{li2004comparing}, that's why it is used by IPFS to create
its distributed file system.

\subsubsection{BitTorrent - File sharing}
\label{tech:sec:ipfs:bt}
BitTorrent \cite{cohen2003incentives} is a P2P file sharing system used
worldwide. In this system, files are divides into very small chucks of data, and
are shared in a peer-to-peer network. Each peer aims to maximize its download
rate by connecting to low latency peers. In BitTorrent's network, peers with
high upload rate will get higher download rate, so the key is balancing the
network bandwidth between downloading and uploading
files\cite{pouwelse2005bittorrent}.

IPFS uses three main features from BitTorrent's protocol\cite{benet2014ipfs}:
\begin{itemize}
\item BitTorrent's data exchange protocol rewards nodes who contribute to the
  network, and punishes the ones who don't.
\item BitTorrent tracks the availability of file chunks, sending the rarest
  first rather than sending the most common ones.
\item IPFS uses PropShare\cite{levin2008bittorrent} bandwidth allocation
  strategy to improve BitTorrent's behavior facing exploitable scenarios.
\end{itemize}

\subsubsection{Git - Version control system}
\label{tech:sec:ipfs:git}
Git is a distributed version control system (\emph{DVCS})\cite{torvalds2010git}.
Git was born in 2005 when the development process of the Linux kernel lost its
version control system. The Linux kernel is one of the biggest free software
projects nowadays, it has a great team of developers behind and the code usually
changes very frequently. In 2002 the team used BitKeeper as VCS since they had a
free license. But in 2005 when this license was over, Linus Torvalds decided to
develop his own VCS\cite{spinellis2005version}.

Git was designed to be scalable and distributed, and the most important factors
that IPFS inherits from Git are: \cite{benet2014ipfs}:

\begin{itemize}
\item Git implements a Merkle Directed Acyclic Graph
  \cite{bleichenbacher1994directed}, an object that reflects changes in a file
  system in a distributed way.
\item Objects are identified by the cryptographic hash of their contents.
\item Version changes only update preferences and add objects. To broadcast
  version changes, git only needs to transfer the new objects and update the
  remote references.
\end{itemize}

\subsubsection{Self-Certified File Systems}
\label{tech:sec:ipfs:scfs}

\section{Ethereum}
\label{tech:sec:ethereum}
\figura{cryptomap.png}{width=0.9\linewidth}{criptomap}{Mapa de las criptos}

Ethereum~\cite{buterin2014ethereum} is a very novel technology that allows the
creation of distributed applications that run in an arbitrary large and
trust-less network of nodes.

\subsection{Blockchain}
\label{tech:sec:ethereum:bc}
Blockchain is the technology behind Bitcoin~\cite{nakamoto2008bitcoin} that
provides a decentralized ledger of 

\subsection{Smart Contracts}
\label{tech:sec:ethereum:sm}
The use of this technology provides significant advantages for our
application, as it does 
%%% Local Variables:
%%% mode: latex
%%% TeX-master: "../Tesis.tex"
%%% End:

\chapter{Platform description}
\label{cha:platform-description}

\begin{FraseCelebre}
  \begin{Frase}
    Try not. Do, or do not. There is no try.
  \end{Frase}
  \begin{Fuente}
    Yoda - The Empire Strikes Back
  \end{Fuente}
\end{FraseCelebre}

We propose a blockchain-enabled decentralized publication system for open
science. It consists of three main components that decentralize and try to
improve three different aspects related to scientific publication:

1) Peer review governance communication is traditionally centralized and
controlled by editors and publishers. Our proposal opens and decentralizes these
communications making the process more transparent.

2) Peer reviewer quality and reliability information is difficult to
predict~\cite{callaham_relationship_2007}, and it is usually hold private by
publishers and journals. The system proposes to open this information through a
decentralized reputation network of peer reviewers over a blockchain.

3) Scientific papers are traditionally obtained or bought from a centralized
publisher. We propose a decentralized network to distribute academic works and
promote free access to science.

These ideas are further discussed in the following sections.

% \section{Decentralized publication system for open science}

% We propose a blockchain-enabled decentralized publication system for open
% science. It consist of three components that decentralizes three different
% aspects of science publication.

% 1) Peer review governance communication is traditionally centralized and
% controlled by editors and publisher. Subsection \ref{workflow} further explore
% the opening and decentralization of this communication with blockchain
% technology.

% 2) Peer reviewer quality and reliability information is difficult to
% predict~\cite{callaham_relationship_2007}, and is usually hold private by
% publishers and journals. The system propose to distribute and open this
% information though a decentralized reputation network of peer reviewers over
% blockchain. Subsection \ref{reputation} further explore this proposal.

% 3) Open access papers are traditionally obtained from a centralized publisher
% infrastructure. Subsection \ref{distributedOA} explores the decentralization
% of this infrastructure.

\section{Transparent Peer Review Governance}
\label{workflow}

The system provides a platform for the peer review process communication, from
paper submission to paper acceptance or rejection. It registers all the
interactions into a blockchain based distributed
ledger. % and provides open access to all relevant content, from the different versions of the paper to the submitted reviews.

The interaction diagram of the system (Figure \ref{InteractionDiagram})
describes the interactions of the supported peer review governance. Following,
this interactions and their implementation are described.

\figura{Uml.png}{width=0.9\linewidth}{umlseq}{Sequence diagram of platform
  interaction}

\begin{LaTeXdescription}
\item[Paper submission] A paper submission is registered by submitting the IPFS
  address of the paper to an Ethereum contract. Then, the Ethereum sender
  address is recorded as the corresponding author, and the submission is
  timestamped in the blockchain.

\item[Reviewer proposal] A journal editor may invite a peer reviewer to review a
  specific paper. The transaction will record the Ethereum address of the
  reviewer and optionally, a deadline to submit the review.

\item[Reviewer acceptance/rejection] An invited reviewer may accept or reject
  the review of a paper. The response will be recorded into the blockchain.

\item[Submit review] A reviewer should make a transaction to deliver the review.
  The transaction will record the acceptance/rejection and the IPFS address of
  the detailed review.

\item[Rate review] A novelty of the system further discussed in Section
  \ref{reputation} is the rating system for reviews. The transaction will record
  the sender address and the rating as well as the rated review and reviewer
  addresses.
\end{LaTeXdescription}

\section{The Peer Review Reputation network}
\label{reputation}

The system proposes the use of a peer review reputation network were the quality
of peer reviews is rated by the authors, editors and reviewers of the system.
The work extends traditional peer review governance with the possibility of
rating the reviews, building a reputation system for
reviewers~\cite{resnick2000reputation}. Reviewers get rewarded for worthy, fair,
an timely reviews, or penalized otherwise.

This network of peer reviewers would enable a better reviewer selection, a fair
recognition of reviewers work and a protection against unfair reviews for
authors. However, it could also rise privacy concerns for both reviewers and
raters~\cite{van1999effect,schaub2016trustless}. We consider these privacy
issues in section \ref{privacy}.

% With this network of rated peer reviews, different metrics of the peer
% reviewers can be provided by the system. Journals can benefit from this
% network by searching for well rated reviewers that respond on time, authors
% can expect shorter review time and forget about unaccountable bad reviews and
% reviewers can get their work publicly recognized. This approach can suppose a
% privacy problem for both reviewers and raters. We contemplate these problems
% and how to solve them in section \ref{PrivacyReviewRating}.

Each time a user of the platform sends a review, it will be rateable. This way,
each reviewer will obtain a score of each one of the reviews he sends, building
the reviewers' reputation system. Initially, the people who can vote are the
authors of the paper, the editors who have assigned the reviewers, and the other
reviewers of the article. But in future implementations, rating a review will be
public.

There are two main implications among others:

\begin{itemize}
  \itbf{The cold start problem for new papers and researchers:} Normally papers
  that are published in journals or conferences of low impact take a long time
  to gain visibility in the scientific community. However, if a reviewer
  reputation system is used, if three reviewers with a high reputation make a
  favorable review for a paper for a little-known journal, it is likely to have
  more visibility since the reputation of the reviewers can positively influence
  the impact of the papers in the scientific community.
  
  \itbf{Pay-per-review:} In most cases nowadays, the reviewers do not obtain any
  economic benefit for carrying out the revision of a paper. By implementing a
  reputation system, pay-per-review dynamics could be formed, in which the most
  reputable reviewers are paid for performing reviews at the request of the
  journal or the authors. This is an incentive for all reviewers of the system
  to make good reviews and gain reputation. In addition, since all interactions
  are in Ethereum, payments and deliveries confirmations can be made through
  personalized smart contracts, eliminating intermediaries, and making the
  process transparent and honest.
 
\end{itemize}


\section{Distributed Open Access infrastructure}
\label{distributedOA}
% \commant{Copied from PEERE Paper.TODO rewrite}
Open Access focuses in the free access to scientific knowledge. While publishers
provide free of charge their Open Access content, their control of the science
dissemination infrastructure allows them to impose certain rules, such as
charging authors unreasonable fees to offer their work as Open
Access~\cite{solomon2012study} (Gold Open Access) or the temporal embargo and
restrictions on the dissemination of the final version (Green Open
access)~\cite{bjork2014anatomy}, among others.

The system proposes a decentralized infrastructure for science publication.
Academic documents - from first drafts to final versions, including peer
reviews- are shared in IPFS, as mentioned on section~\ref{tech:sec:ipfs}. Thus,
the system inherently grants Open Access by the design of its distributed
infrastructure and circumvents the publishers dominant role.

All the documents are first uploaded to IPFS, generating a \emph{Base58}
address. Then the system uses this address to make a transaction in Ethereum,
cryptographically signing the contract with the user's address. Other users that
want to access the paper, only have to inspect the blockchain, retrieve the IPFS
address and download it though a local node or a gateway. This makes
Decentralized Science a truly open-access publication system, were all the
process is public, transparent and trustful.




%%%
%%% Local Variables:
%%% mode: latex
%%% TeX-master: "../Tesis.tex"
%%% End:

\include{Capitulos/Results}
\chapter{Conclusions and future work}

\begin{FraseCelebre}
  \begin{Frase}
    The needs of the many outweigh the needs of the few
  \end{Frase}
  \begin{Fuente}
    Spock - The Wrath of Khan
  \end{Fuente}
\end{FraseCelebre}

% -------------------------------------------------------------------
\section{Cool Section}
% -------------------------------------------------------------------

Lorem ipsum dolor sit amet, consectetur adipiscing elit. Praesent fermentum orci
a justo sagittis, at tincidunt enim luctus. Curabitur imperdiet mauris sed
mattis semper. Aenean augue risus, viverra vel porta a, auctor ac enim. Sed quis
auctor tellus. Suspendisse potenti. Vestibulum nec lectus turpis. Morbi luctus
eros ante, eu consequat magna maximus ut. Donec nec dui sagittis, ornare lorem
a, condimentum odio. Vestibulum ante ipsum primis in faucibus orci luctus et
ultrices posuere cubilia Curae; Nunc quis ipsum eget tellus placerat facilisis.
Curabitur tortor nunc, elementum at imperdiet a, hendrerit id augue. Aenean
vitae lacus eget diam posuere aliquet vel in elit.

Aenean purus est, tempus eget tristique nec, fringilla a enim. Sed in volutpat
eros. Sed commodo congue metus ac aliquam. Quisque auctor dolor libero, vitae
mattis ante luctus sit amet. Aliquam iaculis urna nec lorem rhoncus, commodo
dignissim mi ultrices. Mauris sem augue, luctus ac tincidunt id, sollicitudin at
ipsum. Suspendisse mattis venenatis dolor. Pellentesque velit sem, pulvinar
vitae tortor tempor, blandit aliquet velit. Nunc mattis urna diam, ac maximus
libero molestie non.

Orci varius natoque penatibus et magnis dis parturient montes, nascetur
ridiculus mus. Quisque id egestas tellus. Nunc suscipit ex ac quam pretium, sit
amet vehicula risus hendrerit. Suspendisse faucibus ante metus, sed pulvinar
odio pulvinar condimentum. Donec vel arcu egestas, posuere metus quis, varius
mauris. Suspendisse ut ligula id justo eleifend vestibulum sit amet faucibus
neque. Praesent dignissim risus quis consectetur porta. Maecenas faucibus velit
non pretium ullamcorper. Praesent fringilla pharetra purus. Aenean ullamcorper
nisi gravida sagittis dapibus. Nunc commodo arcu nec cursus venenatis. Integer
vel turpis convallis, feugiat sapien id, euismod arcu. Aliquam sit amet iaculis
lacus. Vestibulum ut ex ac libero efficitur varius at in nisl. Morbi vel posuere
diam, a porttitor turpis.

Pellentesque non justo est. Nunc luctus ullamcorper tincidunt. Aliquam eu nisl a
orci aliquam cursus. Mauris id sollicitudin mauris. Suspendisse quis dolor id
magna porttitor mollis eu sit amet diam. Donec a elementum nibh. Sed egestas id
sapien nec aliquet. Vivamus porta dignissim bibendum. Donec at odio volutpat,
vestibulum elit at, fringilla ipsum. Donec vehicula lectus efficitur est
fermentum, ac tincidunt quam maximus. Proin lectus sem, sodales quis nunc ut,
maximus ullamcorper erat. Mauris lobortis justo quis malesuada viverra. Sed
suscipit, elit quis tincidunt condimentum, ipsum augue ornare velit, sit amet
fermentum leo orci ac risus. Vestibulum elementum viverra porta.

Sed gravida, risus nec scelerisque egestas, metus elit maximus leo, sit amet
lacinia erat purus at mauris. Aliquam eget libero velit. Proin luctus risus et
maximus efficitur. Praesent eget cursus ipsum. In hac habitasse platea dictumst.
Duis fringilla purus eu enim hendrerit, ut gravida tortor scelerisque. Donec
quis pellentesque ex, in auctor turpis. Ut non turpis purus.

Duis porttitor turpis purus, eget volutpat diam posuere eget. Morbi porttitor
risus quis tortor pellentesque varius. Nullam eu odio a augue tincidunt cursus.
Proin semper sapien augue, sit amet maximus turpis vestibulum ac. Quisque semper
justo nunc, sed efficitur augue ullamcorper ac. Praesent egestas eget neque quis
consectetur. Ut quis nulla rutrum, placerat leo nec, euismod quam. Phasellus
dapibus ligula vitae lacus lacinia blandit. Morbi vel ligula iaculis, aliquet ex
ut, scelerisque sem. Donec rutrum lacus quis odio vulputate ultrices. Quisque
ullamcorper rhoncus mauris, ac vestibulum magna blandit et. Suspendisse eu diam
rhoncus, sagittis quam vitae, maximus est. Donec augue diam, euismod a elit ac,
imperdiet accumsan nunc.

Aenean quis metus sed urna fringilla scelerisque. Proin tellus lectus, laoreet
tincidunt commodo eget, euismod sit amet nisi. Morbi eu massa eu arcu lobortis
malesuada. Orci varius natoque penatibus et magnis dis parturient montes,
nascetur ridiculus mus. Donec ut enim vel lacus eleifend suscipit. Curabitur id
ex vitae leo tempor elementum. In vestibulum mi eu ligula sagittis, eu efficitur
est auctor. Morbi ornare molestie rutrum. Aliquam sit amet fermentum enim. Fusce
sed tempus sem. In elementum dolor nec justo tempus hendrerit. Cras sit amet
cursus est, sit amet porta lacus. Nulla pretium at dolor quis volutpat.
Pellentesque condimentum ultricies urna, sed aliquam ipsum. Praesent consequat
faucibus massa id porta. Nullam nibh purus, maximus eu tristique ut, convallis
sit amet sapien.

Integer ac nisi leo. Quisque sollicitudin eros lorem, in suscipit orci congue
eget. Suspendisse gravida augue nec faucibus interdum. Integer blandit ligula
nec ex dapibus sodales. Aenean accumsan ante nibh, ac varius nisl pharetra in.
Duis mollis convallis neque, vitae finibus dui tincidunt ac. Aliquam a leo sed
elit porttitor dignissim blandit sit amet mauris. Proin viverra id ex quis
maximus. Etiam ultrices tristique faucibus. Donec egestas lorem enim, vel rutrum
erat interdum eu. Vestibulum turpis libero, aliquet non neque nec, placerat
tempus nulla.

Phasellus accumsan lacus ut enim rutrum posuere. In sit amet enim mi. Nulla et
fringilla justo. Etiam ut posuere velit, non fringilla risus. Praesent mattis
diam nec convallis finibus. Donec malesuada felis eu consectetur volutpat. Nulla
mattis neque eu arcu fermentum, ac accumsan quam placerat. Maecenas iaculis
blandit leo, ut semper orci vestibulum et. Quisque ut lorem pretium, condimentum
lorem vel, volutpat nisl. Maecenas ut enim sed elit maximus aliquam eget in
diam. Donec vulputate nunc sed diam accumsan tempus.

Phasellus mollis tempor lectus vitae pretium. Praesent ullamcorper suscipit est
eleifend aliquet. Nulla nec diam consectetur, venenatis arcu non, finibus odio.
Ut vel nisi condimentum, imperdiet enim id, ultrices felis. Duis congue dui
sapien, eu mattis ipsum mollis at. Donec a ex a orci tincidunt fermentum.
Aliquam pretium non orci nec pharetra. Donec tellus erat, maximus et
sollicitudin eget, elementum a mauris. Phasellus tempus mauris vitae tortor
elementum, eleifend viverra nisl luctus.

Aliquam in est sed velit elementum dictum. Phasellus auctor est sit amet egestas
vulputate. Nunc volutpat nibh lacus, eu dignissim libero gravida sit amet. Sed
accumsan dui sit amet leo pretium, id porttitor nisl fringilla. Mauris vehicula
pretium cursus. Integer bibendum tellus a ante eleifend tempor. Phasellus vitae
efficitur felis, a vulputate nulla. In finibus massa ut ante consequat auctor.
Phasellus elementum odio eget viverra lobortis. Lorem ipsum dolor sit amet,
consectetur adipiscing elit. Nam et quam et tellus ullamcorper consequat non non
mauris. Sed molestie ut turpis quis consectetur. Ut hendrerit justo eget orci
faucibus dignissim. Aenean ut viverra nulla. Proin vestibulum vehicula dapibus.

Cras posuere laoreet dui sit amet bibendum. Phasellus mollis euismod sapien eget
iaculis. Donec vel commodo ante. Sed vitae elit eget felis suscipit rhoncus. Nam
lacus leo, porttitor et vestibulum sodales, convallis eget sapien. Fusce quam
tellus, finibus eget feugiat at, ultrices sed risus. Cras ac semper odio.

Cras egestas rhoncus lorem ac cursus. Sed ut ipsum ut urna commodo tincidunt sit
amet id nulla. Integer purus tortor, iaculis vel nisi sit amet, commodo varius
diam. Etiam accumsan, diam sit amet posuere laoreet, felis mauris tincidunt
ante, vitae hendrerit nibh sapien in sem. Suspendisse rhoncus a nunc eget
ornare. Quisque posuere mauris nunc, ac convallis nisi condimentum vel. Integer
pulvinar neque at ligula interdum, ac blandit augue ultricies. Aliquam fermentum
enim in metus volutpat ullamcorper. Nam rutrum posuere mi, sit amet molestie
erat convallis eget. Vestibulum eu dictum lorem. Ut commodo cursus nisl, non
elementum quam ullamcorper eget.

Vivamus in tincidunt enim, quis tempus metus. Nam eu tristique nibh, sit amet
consectetur quam. Nulla nisi erat, tristique eget urna sit amet, viverra congue
metus. Cras eu nisl vel diam lobortis tempor non at tortor. Morbi maximus tellus
placerat elementum elementum. Praesent nec dignissim dolor. Sed eget purus
tortor. Nunc quis ullamcorper lacus.

Aenean quis lectus congue, vehicula leo porta, viverra velit. Suspendisse
potenti. Etiam fermentum tempus enim. Integer ac pharetra odio, eget facilisis
velit. Praesent auctor risus nec mauris imperdiet maximus. Mauris venenatis nisi
vitae aliquam fermentum. Ut volutpat bibendum tincidunt. Nam quis risus at orci
porttitor blandit ac a nulla. Ut ipsum quam, dapibus eu ligula id, molestie
accumsan elit. Nam rhoncus sem sit amet elit finibus, posuere porttitor augue
facilisis.

%%%
%%% Local Variables:
%%% mode: latex
%%% TeX-master: "../Tesis.tex"
%%% End:

%\include{...}
%\include{...}
%\include{...}
%\include{...}

% Apéndices
%\appendix
%\include{Apendices/ParteApendices}
%%---------------------------------------------------------------------
%
%                          Apéndice 1
%
%---------------------------------------------------------------------


\chapter{Anexo 1: Reuniones del equipo}
\label{ap1:Reuniones}

\begin{FraseCelebre}
\begin{Frase}
...
\end{Frase}
\begin{Fuente}
...
\end{Fuente}
\end{FraseCelebre}

\begin{resumen}
...
\end{resumen}

\section{Reunión del 07 de Septiembre de 2017}

En la primera reunión del equipo se hicieron las presentaciones de los integrantes, y se discutieron las posibles ideas que se podrían implementar como proyecto en la \textit{hackathon}.
El tema principal sobre el que se discutía era el impacto social del proyecto, y que las métricas de la \textit{hackathon} así lo exigían.

Realizamos una tormenta de ideas en la que surgieron las siguientes:

\begin{itemize}
  \item \textbf{Plataforma de publicación de artículos académicos distribuida}: Implementar un sistema de publicación de artículos para compartir a través de la comunidad científica utilizando IPFS. Esta plataforma pretende eliminar los costes para el acceso a los artículos que imponen las empresas que se encargan de publicarlos y se benefician por ello. Implementar una plataforma totalmente descentralizada para compartir los artículos de divulgación científica conseguirá que el conocimiento de la investigación académica sea público y accesible por todos.

  \item \textbf{Wikipedia distribuida con modelos de gobernanza}: La idea de este proyecto inicialmente era descentralizar la plataforma de Wikipedia a través de IFPS y añadir algún modelo de gobernanza y de reputación para las revisiones de los artículos. EL problema es que es un proyecto muy complejo para implementarlo en sólo un mes, y haría falta un equipo bastante grande y la colaboración de la propia Wikipedia para llevarlo a cabo.

  \item \textbf{Aplicación de contactos para homosexuales en países donde son colectivos reprimidos}: En países como Rusia, los colectivos LGBT son reprimidos hasta el punto de que expresar su sexualidad puede ser un peligro para su seguridad personal. Esta idea trataba de poner en contacto de la manera más anónima y discreta posible a esas personas sin exponerse a los riesgos que ello conlleva.

  \item \textbf{ONG distribuida}: Esta plataforma pretendía ofrecer una bolsa de dinero en la que las personas iban realizando donaciones. Cada semana los donantes votaban dónde se iban a invertir el dinero mediante un sistema de votos.

  \item \textbf{Plataforma de intercambio de conocimientos de programación distribuida}: Stack Exchange es una de las web más importantes en la comunidad informática. Esta solución propone una altenativa totalmente distribuida mediante blockchain.

  \item \textbf{Plataforma de toma de decisiones distribuida}: La toma de decisiones en comunidades reprimidas es bastante dificil. Mediante una aplicación de toma de decisiones en blockchain (como la que tiene Loomio), se pueden ofrecer una herramienta para que estas personas en riesgo de exclusión se hagan oir.

  \item \textbf{Plataforma de crowdfunding para \textit{wistleblowers}}: El problema de las plataformas de crowdfunding es que una vez que se financia el proyecto, el usuario sólo puede ver el final del producto esperando que lo que ha financiado sea como promenten los desarrolladores. Esta plataforma propondría una alternativa con varios entregables en función del dinero que se vaya consiguiendo.


\end{itemize}

...

% Variable local para emacs, para  que encuentre el fichero maestro de
% compilación y funcionen mejor algunas teclas rápidas de AucTeX
%%%
%%% Local Variables:
%%% mode: latex
%%% TeX-master: "../Tesis.tex"
%%% End:

\section{Reunión del 08 de Septiembre de 2017}

Una vez que el equipo ha decidido el proyecto que vamos a afrontar, nos reunimos para ir decidiendo poco a poco las funcionalidades que habría de tener nuestra plataforma. Algunas de ellas son:

%\include{...}
%\include{...}
%\include{...}

\backmatter

%
% Bibliograf_a
%

%---------------------------------------------------------------------
%
%                      configBibliografia.tex
%
%---------------------------------------------------------------------
%
% bibliografia.tex
% Copyright 2009 Marco Antonio Gomez-Martin, Pedro Pablo Gomez-Martin
%
% This file belongs to the TeXiS manual, a LaTeX template for writting
% Thesis and other documents. The complete last TeXiS package can
% be obtained from http://gaia.fdi.ucm.es/projects/texis/
%
% Although the TeXiS template itself is distributed under the
% conditions of the LaTeX Project Public License
% (http://www.latex-project.org/lppl.txt), the manual content
% uses the CC-BY-SA license that stays that you are free:
%
%    - to share & to copy, distribute and transmit the work
%    - to remix and to adapt the work
%
% under the following conditions:
%
%    - Attribution: you must attribute the work in the manner
%      specified by the author or licensor (but not in any way that
%      suggests that they endorse you or your use of the work).
%    - Share Alike: if you alter, transform, or build upon this
%      work, you may distribute the resulting work only under the
%      same, similar or a compatible license.
%
% The complete license is available in
% http://creativecommons.org/licenses/by-sa/3.0/legalcode
%
%---------------------------------------------------------------------
%
% Fichero  que  configura  los  par_metros  de  la  generaci_n  de  la
% bibliograf_a.  Existen dos  par_metros configurables:  los ficheros
% .bib que se utilizan y la frase c_lebre que aparece justo antes de la
% primera referencia.
%
%---------------------------------------------------------------------


%%%%%%%%%%%%%%%%%%%%%%%%%%%%%%%%%%%%%%%%%%%%%%%%%%%%%%%%%%%%%%%%%%%%%%
% Definici_n de los ficheros .bib utilizados:
% \setBibFiles{<lista ficheros sin extension, separados por comas>}
% Nota:
% Es IMPORTANTE que los ficheros est_n en la misma l_nea que
% el comando \setBibFiles. Si se desea utilizar varias l_neas,
% terminarlas con una apertura de comentario.
%%%%%%%%%%%%%%%%%%%%%%%%%%%%%%%%%%%%%%%%%%%%%%%%%%%%%%%%%%%%%%%%%%%%%%
\setBibFiles{%
latex,otros,nuestros}

%%%%%%%%%%%%%%%%%%%%%%%%%%%%%%%%%%%%%%%%%%%%%%%%%%%%%%%%%%%%%%%%%%%%%%
% Definici_n de la frase c_lebre para el cap_tulo de la
% bibliograf_a. Dentro normalmente se querr_ hacer uso del entorno
% \begin{FraseCelebre}, que contendr_ a su vez otros dos entornos,
% un \begin{Frase} y un \begin{Fuente}.
%
% Nota:
% Si no se quiere cita, se puede eliminar su definici_n (en la
% macro setCitaBibliografia{} ).
%%%%%%%%%%%%%%%%%%%%%%%%%%%%%%%%%%%%%%%%%%%%%%%%%%%%%%%%%%%%%%%%%%%%%%
\setCitaBibliografia{
\begin{FraseCelebre}
  \begin{Frase}
    You act, and you know why you act, but you don't know why you know that you know what you do?
\end{Frase}
\begin{Fuente}
  The name of the rose - Umberto eco
\end{Fuente}
\end{FraseCelebre}
}

%%
%% Creamos la bibliografia
%%
\makeBib

% Variable local para emacs, para  que encuentre el fichero maestro de
% compilaci_n y funcionen mejor algunas teclas r_pidas de AucTeX

%%%
%%% Local Variables:
%%% mode: latex
%%% TeX-master: "../Tesis.tex"
%%% End:


%
% _ndice de palabras
%

% S_lo  la   generamos  si  est_   declarada  \generaindice.  Consulta
% TeXiS.sty para m_s informaci_n.

% En realidad, el soporte para la generaci_n de _ndices de palabras
% en TeXiS no est_ documentada en el manual, porque no ha sido usada
% "en producci_n". Por tanto, el fichero que genera el _ndice
% *no* se incluye aqu_ (est_ comentado). Consulta la documentaci_n
% en TeXiS_pream.tex para m_s informaci_n.
\ifx\generaindice\undefined
\else
%\include{TeXiS/TeXiS_indice}
\fi

%
% Lista de acr_nimos
%

% S_lo  lo  generamos  si  est_ declarada  \generaacronimos.  Consulta
% TeXiS.sty para m_s informaci_n.


\ifx\generaacronimos\undefined
\else
%---------------------------------------------------------------------
%
%                        TeXiS_acron.tex
%
%---------------------------------------------------------------------
%
% TeXiS_acron.tex
% Copyright 2009 Marco Antonio Gomez-Martin, Pedro Pablo Gomez-Martin
%
% This file belongs to TeXiS, a LaTeX template for writting
% Thesis and other documents. The complete last TeXiS package can
% be obtained from http://gaia.fdi.ucm.es/projects/texis/
%
% This work may be distributed and/or modified under the
% conditions of the LaTeX Project Public License, either version 1.3
% of this license or (at your option) any later version.
% The latest version of this license is in
%   http://www.latex-project.org/lppl.txt
% and version 1.3 or later is part of all distributions of LaTeX
% version 2005/12/01 or later.
%
% This work has the LPPL maintenance status `maintained'.
%
% The Current Maintainers of this work are Marco Antonio Gomez-Martin
% and Pedro Pablo Gomez-Martin
%
%---------------------------------------------------------------------
%
% Contiene  los  comandos  para  generar  el listado de acr_nimos
% documento.
%
%---------------------------------------------------------------------
%
% NOTA IMPORTANTE:  para que la  generaci_n de acr_nimos  funcione, al
% menos  debe  existir  un  acr_nimo   en  el  documento.  Si  no,  la
% compilaci_n  del   fichero  LaTeX  falla  con   un  error  "extra_o"
% (indicando  que  quiz_  falte  un \item).   Consulta  el  comentario
% referente al paquete glosstex en TeXiS_pream.tex.
%
%---------------------------------------------------------------------


% Redefinimos a espa_ol  el t_tulo de la lista  de acr_nimos (Babel no
% lo hace por nosotros esta vez)

\def\listacronymname{Lista de acr_nimos}

% Para el glosario:
% \def\glosarryname{Glosario}

% Si se  va a generar  la tabla de  contenidos (el _ndice  habitual) y
% tambi_n vamos a  generar la lista de acr_nimos  (ambas decisiones se
% toman en  funci_n de  la definici_n  o no de  un par  de constantes,
% puedes consultar config.tex  para m_s informaci_n), entonces metemos
% en la  tabla de contenidos una  entrada para marcar  la p_gina donde
% est_ el _ndice de palabras.

\ifx\generatoc\undefined
\else
   \addcontentsline{toc}{chapter}{\listacronymname}
\fi


% Generamos la lista de acr_nimos (en realidad el _ndice asociado a la
% lista "acr" de GlossTeX)

\printglosstex(acr)

% Variable local para emacs, para  que encuentre el fichero maestro de
% compilaci_n y funcionen mejor algunas teclas r_pidas de AucTeX

%%%
%%% Local Variables:
%%% mode: latex
%%% TeX-master: "../Tesis.tex"
%%% End:

\fi

%
% Final
%
%---------------------------------------------------------------------
%
%                      fin.tex
%
%---------------------------------------------------------------------
%
% fin.tex
% Copyright 2009 Marco Antonio Gomez-Martin, Pedro Pablo Gomez-Martin
%
% This file belongs to the TeXiS manual, a LaTeX template for writting
% Thesis and other documents. The complete last TeXiS package can
% be obtained from http://gaia.fdi.ucm.es/projects/texis/
%
% Although the TeXiS template itself is distributed under the
% conditions of the LaTeX Project Public License
% (http://www.latex-project.org/lppl.txt), the manual content
% uses the CC-BY-SA license that stays that you are free:
%
%    - to share & to copy, distribute and transmit the work
%    - to remix and to adapt the work
%
% under the following conditions:
%
%    - Attribution: you must attribute the work in the manner
%      specified by the author or licensor (but not in any way that
%      suggests that they endorse you or your use of the work).
%    - Share Alike: if you alter, transform, or build upon this
%      work, you may distribute the resulting work only under the
%      same, similar or a compatible license.
%
% The complete license is available in
% http://creativecommons.org/licenses/by-sa/3.0/legalcode
%
%---------------------------------------------------------------------
%
% Contiene la última página
%
%---------------------------------------------------------------------


% Ponemos el marcador en el PDF al nivel adecuado, dependiendo
% de su hubo partes en el documento o no (si las hay, queremos
% que aparezca "al mismo nivel" que las partes.
\ifpdf
\ifx\tienePartesTeXiS\undefined
   \pdfbookmark[0]{Fin}{fin}
\else
   \pdfbookmark[-1]{Fin}{fin}
\fi
\fi

\thispagestyle{empty}\mbox{}

\vspace*{4cm}

\small

\hfill \emph{Until they become conscious they will never rebel,}

\hfill \emph{and until after they have rebelled they cannot become conscious.}

\hfill

\hfill \emph{1984}

\hfill \emph{George Orwell}


\hfill \vspace{1cm}

\hfill \emph{These people don’t see that if you encourage totalitarian methods,}

\hfill \emph{the time may come when they will be used against you instead of for
  you.}

\hfill

\hfill \emph{Animal Farm}

\hfill \emph{George Orwell}

\newpage
\thispagestyle{empty}\mbox{}

\newpage

% Variable local para emacs, para  que encuentre el fichero maestro de
% compilación y funcionen mejor algunas teclas rápidas de AucTeX

%%%
%%% Local Variables:
%%% mode: latex
%%% TeX-master: "../Tesis.tex"
%%% End:


\end{document}
